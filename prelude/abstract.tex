% abstract.tex
%
% This file has the abstract for the withesis style documentation
%
% Eric Benedict, Aug 2000
%
% It is provided without warranty on an AS IS basis.

%\noindent       % Don't indent this paragraph.
%This is not a thesis or dissertation and Master \TeX nician is not a
%degree granted at the University of Wisconsin-Madison.

%\vspace*{0.5em}
%\noindent       % Don't indent this paragraph.
%This explains the basics for using \LaTeX\ to typeset a dissertation,
%thesis or masters project or preliminary report for the University of 
%Wisconsin-Madison. Chapter
%1 talks briefly about the thesis formatting at UW-Madison.  Chapter 2 gives
%an overview of the ``essentials'' of \LaTeX{} and was written by Jon Warbrick.
%Chapter 3 talks about figures and tables and what a {\em float} is.  Chapter 4
%briefly introduces the {\sc Bib}\TeX{} program.  And finally, Chapter 5 discusses
%some of the details for using the {\tt withesis} style file.  The material in
%Chapters 2-4 basically are a review of fundamental \LaTeX{} usage and form
%a reasonable basic tutorial.%

%\vspace*{0.5em}
%\noindent       % Don't indent this paragraph.
%The style discussed in this manual was originally written by Dave Kraynie and
%edited by James Darrell McCauley as the {\tt puthesis} style for Purdue
%University's theses.  This style was modified to form the {\tt withesis} style. This
%manual is largely based on a similar manual by James Darrell McCauley and Scott Hucker.
%Permission to use, copy, modify and distribute this software and its documentation
%for any purpose and without fee is here by granted.  This software and its documentation
%is provided ``as is'' without any express or implied warranty.

Cloud outage can result in bad user experiences for cloud tenants and revenue loss to the provider. 
This makes cloud network diagnostic solutions invaluable.
Despite the various existing network diagnostic solutions, few of them are designed 
specifically for cloud networks. Current state-of-the-art cloud network diagnosis falls short 
in three aspects: (1) there is no clear way to distinguish whether an observed problem is in the 
tenant's virtual network or in the provider's infrastructure. As a result, 
the interaction between tenants and the provider leads to a longer problem-solving time 
and higher maintenance costs; (2) for cloud tenants, there are only rudimentary troubleshooting 
tools (e.g., ping, VM monitoring) that can be deployed. However, diagnosing a distributed 
system with these tools depends heavily on skill and experience, which is not always 
feasible for tenants; (3) for the cloud provider, new trends such as network function 
virtualization make the infrastructure more complex than the traditional network, 
which could lead to new problems arising. Thus, the provider requires new diagnostic tools to help
cover this range of problems.

In this thesis, we design two systems for cloud network diagnosis: (A) \emph{VND: a Virtual Network 
Diagnostic Service.} VND is a service offered by the provider to its tenants.
Using VND, a tenant can determine whether a problem is in its virtual network or not; 
VND's interfaces also simplify tenants' troubleshooting operations. A tenant could use VND to 
collect, parse and query its packet traces. Here, the trace collection cooperates within
%interface is designed in 
the tenant's view of its own virtual network without exposing the cloud infrastructure. 
Trace parse and query interfaces are design to ease the tenant's troubleshooting operations. 
VND provides a SQL interface for tenants to perform diagnosis. We show several typical 
network diagnostic use cases where troubleshooting solutions can be easily implemented using VND. 
We also measure VND's overhead and show its feasibility.
(B) \emph{PerfSight: Performance Diagnosis for Software Data Planes.}
Increasingly, modern network data planes have 
complex software involved in packet processing (e.g., virtual switches, VM hypervisors and software middleboxes).
%For example, network virtualization introduces virtual switches and VM hypervisors; network function virtualization introduces various software middleboxes. 
We refer to these software parts as the software data plane. We argue that there are at least three new
classes of performance problems that arise in software data planes: bottlenecks, contentions and bugs. 
We propose 
a system named PerfSight to target these three problems. PerfSight instruments the software 
data plane, gathers basic statistics (e.g., packet count, byte count, I/O time) and analyzes the 
statistics comprehensively. We obtained two key insights by running PerfSight: (1) packet drop is the best 
indicator of bottlenecks, and location of packet drop can give information on the resources in contention 
(e.g. CPU, network bandwidth); (2) software middlebox's states can be defined by basic statistics, 
and these states propagate in the network in certain patterns. These patterns can be used 
to infer which middlebox has performance bugs. 
%We also some case studies on how PerfSight can help with cloud management.
Our evaluation shows \Name introduces little overhead to the existing system, and thus it is 
feasible to deploy.

Together, we believe VND and \Name provide diagnostic solutions to both tenants and the provider. 
They form an integral basis for cloud network diagnosis.
%make a complement for cloud network diagnosis. 
