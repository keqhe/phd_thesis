% abstract.tex
%
% This file has the abstract for the withesis style documentation
%
% Eric Benedict, Aug 2000
%
% It is provided without warranty on an AS IS basis.

%\noindent       % Don't indent this paragraph.
%This is not a thesis or dissertation and Master \TeX nician is not a
%degree granted at the University of Wisconsin-Madison.

%\vspace*{0.5em}
%\noindent       % Don't indent this paragraph.
%This explains the basics for using \LaTeX\ to typeset a dissertation,
%thesis or masters project or preliminary report for the University of 
%Wisconsin-Madison. Chapter
%1 talks briefly about the thesis formatting at UW-Madison.  Chapter 2 gives
%an overview of the ``essentials'' of \LaTeX{} and was written by Jon Warbrick.
%Chapter 3 talks about figures and tables and what a {\em float} is.  Chapter 4
%briefly introduces the {\sc Bib}\TeX{} program.  And finally, Chapter 5 discusses
%some of the details for using the {\tt withesis} style file.  The material in
%Chapters 2-4 basically are a review of fundamental \LaTeX{} usage and form
%a reasonable basic tutorial.%

%\vspace*{0.5em}
%\noindent       % Don't indent this paragraph.
%The style discussed in this manual was originally written by Dave Kraynie and
%edited by James Darrell McCauley as the {\tt puthesis} style for Purdue
%University's theses.  This style was modified to form the {\tt withesis} style. This
%manual is largely based on a similar manual by James Darrell McCauley and Scott Hucker.
%Permission to use, copy, modify and distribute this software and its documentation
%for any purpose and without fee is here by granted.  This software and its documentation
%is provided ``as is'' without any express or implied warranty.

Cloud outage can result in bad user experiences for cloud tenants and revenue loss to the provider. 
This makes cloud network diagnostic solutions invaluable.
Despite the various existing network diagnostic solutions, few of them are designed 
specifically for cloud networks. Current state-of-the-art cloud network diagnosis falls short 
in three aspects: (1) there is no clear way to distinguish whether an observed problem is in the 
tenant's virtual network or in the provider's infrastructure. As a result, 
the interaction between tenants and the provider leads to a longer problem-solving time 
and higher maintenance costs; (2) for cloud tenants, there are only rudimentary troubleshooting 
tools (e.g., ping, VM monitoring) that can be deployed. However, diagnosing a distributed 
system with these tools depends heavily on skill and experience, which is not always 
feasible for tenants; (3) for the cloud provider, new trends such as network function 
virtualization make the infrastructure more complex than the traditional network, 
which could lead to new problems arising. Thus, the provider requires new diagnostic tools to help
cover this range of problems.

