% prelude.tex
%   - titlepage
%   - dedication
%   - acknowledgments
%   - table of contents, list of tables and list of figures
%   - nomenclature
%   - abstract
%============================================================================


\clearpage\pagenumbering{roman}  % This makes the page numbers Roman (i, ii, etc)



% TITLE PAGE
%   - define \title{} \author{} \date{}
\title{Towards Systematic Diagnosis for Cloud Networks}
\author{Wenfei Wu}
\date{}
%   - The default degree is ``Doctor of Philosophy''
%     (unless the document style msthesis is specified
%      and then the default degree is ``Master of Science'')
%     Degree can be changed using the command \degree{}
\degree{Doctor of Philosophy}
%   - The default is dissertation, unless the document style
%     msthesis was specified in which case it becomes thesis.
%     If msthesis is specified for the MS margins, you can
%     still have a dissertation if you specify \disseration
%\disseration
%   - for a masters project report, specify \project
%\project
%   - for a preliminary report, specify \prelim
%\prelim
%   - for a masters thesis, specify \thesis
\thesis
%   - The default department is ``Electrical Engineering''
%     The department can be changed using the command \department{}
\department{Computer Sciences}
%   - once the above are defined, use \maketitle to generate the titlepage
\oralexamdate{9/29/2015}
\committeeone{Srinivasa A. Akella, Associate Professor, Computer Science}
\committeetwo{Remzi H. Arpaci-Dusseau, Professor, Computer Science}
\committeethree{Michael M. Swift, Associate Professor, Computer Science}
\committeefour{Shan Lu, Associate Professor, Computer Science, University of Chicago}
\committeefive{Xinyu Zhang, Assistant Professor, Electrical and Computer Engineering}
\date{2015}


\maketitle

% COPYRIGHT PAGE
%   - To include a copyright page use \copyrightpage
\copyrightpage

% DEDICATION
\begin{dedication}
\emph{To my parents.}
\end{dedication}

% ACKNOWLEDGMENTS
\begin{acknowledgments}

Pursuing a Ph.D. at the University of Wisconsin-Madison has been one of the most wonderful experiences of my life. I would like to take this opportunity to thank the people who guided, inspired, and accompanied me to go through this memorable experience.
 
First and foremost, I would like to thank my adviser Professor Aditya Akella. In addition to teaching me how to conduct research, his enthusiasm and focus on research set a good example for me. I am especially grateful for his support and patience during my difficulties research.
 
I would like to thank the members of my thesis committee, Remzi Arpaci-Dusseau, Michael Swift, Shan Lu, and Xinyu Zhang. Their valuable suggestions and comments greatly improved this thesis.
 
I would like to thank my mentors during my several internships, Chuanxiong GPO, Yoshio Turner, Michael Schlansker, Anees Shaikh, Guohui Wang, Alex Tessmer, and Li Erran Li. I appreciate the industrial internship 
experience I gained while working with them, particularly the chance to learn about and solve practical problems.
 
I would like to thank my fellow graduate students and postdocs, Ashok Anand, Theophuius Benson, Shan-Hsiang Shen, Aaron Gember-Jacobson, Robert Grandl, Junaid Khalid, Chaithan Prakash, David Tran-Lam, Raajay Vishwanathan, Xiaoyang Gao, Ramakrishnan Durairajan, and Brent Stephens. Working with these brilliant people is a great gift to me. I am also quite fortunate to have great friends around me, who constantly supported and cheered me: Keqiang He, Linhai Song, Yizheng Chen, Ao Ma, Yupu Zhang, Lanyue Lu, and Suli Yang.
 
Last but not least, I would like to thank my family. My parents always stand behind me with their love, support, and encouragement, and have been great sources of inspiration throughout the highs and lows of my Ph.D. work. I would also like to thank my relatives, who inspired and supported me.

\end{acknowledgments}

% CONTENTS, TABLES, FIGURES
\tableofcontents
\listoftables
\listoffigures

% NOMENCLATURE
%\begin{nomenclature}
%\begin{description}
%\item{\makebox[0.75in][l]{\TeX}}
%       \parbox[t]{5in}{a typesetting system by Donald Knuth~\cite{knuth}.  It
%       also refers to the ``plain'' format.  The proper pronounciation
%       rhymes with ``heck'' and ``peck'' and does not sound like
%       ``hex'' or ``Rex.''\\}
%
%\item{\makebox[0.75in][l]{\LaTeX}}  
%        \parbox[t]{5in}{a set of \TeX{} macros originally written by Leslie 
%        Lamport~\cite{lamport}.  The proper pronunciation is 
%        {\tt l\={a}$\cdot$tek'} and not {\tt l\={a}'$\cdot$teks} (see above).\\}
%
%\item{\makebox[0.75in][l]{{\sc Bib}\TeX}} 
%         \parbox[t]{5in}{a bibliography generation program by Oren 
%                Patashnik~\cite{lamport}
%                that can be used with either plain \TeX{} or \LaTeX{}.\\}
%
%\item{\makebox[0.75in][l]{$C_1$}} Constant 1
%
%\item{\makebox[0.75in][l]{$V$}}    Voltage 
%
%\item{\makebox[0.75in][l]{\$}}     US Dollars
%\end{description}
%\end{nomenclature}


\advisorname{Aditya Akella}
\advisortitle{Associate Professor}
% ABSTRACT
%\begin{umiabstract}
%  % abstract.tex
%
% This file has the abstract for the withesis style documentation
%
% Eric Benedict, Aug 2000
%
% It is provided without warranty on an AS IS basis.

%\noindent       % Don't indent this paragraph.
%This is not a thesis or dissertation and Master \TeX nician is not a
%degree granted at the University of Wisconsin-Madison.

%\vspace*{0.5em}
%\noindent       % Don't indent this paragraph.
%This explains the basics for using \LaTeX\ to typeset a dissertation,
%thesis or masters project or preliminary report for the University of 
%Wisconsin-Madison. Chapter
%1 talks briefly about the thesis formatting at UW-Madison.  Chapter 2 gives
%an overview of the ``essentials'' of \LaTeX{} and was written by Jon Warbrick.
%Chapter 3 talks about figures and tables and what a {\em float} is.  Chapter 4
%briefly introduces the {\sc Bib}\TeX{} program.  And finally, Chapter 5 discusses
%some of the details for using the {\tt withesis} style file.  The material in
%Chapters 2-4 basically are a review of fundamental \LaTeX{} usage and form
%a reasonable basic tutorial.%

%\vspace*{0.5em}
%\noindent       % Don't indent this paragraph.
%The style discussed in this manual was originally written by Dave Kraynie and
%edited by James Darrell McCauley as the {\tt puthesis} style for Purdue
%University's theses.  This style was modified to form the {\tt withesis} style. This
%manual is largely based on a similar manual by James Darrell McCauley and Scott Hucker.
%Permission to use, copy, modify and distribute this software and its documentation
%for any purpose and without fee is here by granted.  This software and its documentation
%is provided ``as is'' without any express or implied warranty.

Cloud outage can result in bad user experiences for cloud tenants and revenue loss to the provider. 
This makes cloud network diagnostic solutions invaluable.
Despite the various existing network diagnostic solutions, few of them are designed 
specifically for cloud networks. Current state-of-the-art cloud network diagnosis falls short 
in three aspects: (1) there is no clear way to distinguish whether an observed problem is in the 
tenant's virtual network or in the provider's infrastructure. As a result, 
the interaction between tenants and the provider leads to a longer problem-solving time 
and higher maintenance costs; (2) for cloud tenants, there are only rudimentary troubleshooting 
tools (e.g., ping, VM monitoring) that can be deployed. However, diagnosing a distributed 
system with these tools depends heavily on skill and experience, which is not always 
feasible for tenants; (3) for the cloud provider, new trends such as network function 
virtualization make the infrastructure more complex than the traditional network, 
which could lead to new problems arising. Thus, the provider requires new diagnostic tools to help
cover this range of problems.


%\end{umiabstract}

\begin{abstract}
  % abstract.tex
%
% This file has the abstract for the withesis style documentation
%
% Eric Benedict, Aug 2000
%
% It is provided without warranty on an AS IS basis.

%\noindent       % Don't indent this paragraph.
%This is not a thesis or dissertation and Master \TeX nician is not a
%degree granted at the University of Wisconsin-Madison.

%\vspace*{0.5em}
%\noindent       % Don't indent this paragraph.
%This explains the basics for using \LaTeX\ to typeset a dissertation,
%thesis or masters project or preliminary report for the University of 
%Wisconsin-Madison. Chapter
%1 talks briefly about the thesis formatting at UW-Madison.  Chapter 2 gives
%an overview of the ``essentials'' of \LaTeX{} and was written by Jon Warbrick.
%Chapter 3 talks about figures and tables and what a {\em float} is.  Chapter 4
%briefly introduces the {\sc Bib}\TeX{} program.  And finally, Chapter 5 discusses
%some of the details for using the {\tt withesis} style file.  The material in
%Chapters 2-4 basically are a review of fundamental \LaTeX{} usage and form
%a reasonable basic tutorial.%

%\vspace*{0.5em}
%\noindent       % Don't indent this paragraph.
%The style discussed in this manual was originally written by Dave Kraynie and
%edited by James Darrell McCauley as the {\tt puthesis} style for Purdue
%University's theses.  This style was modified to form the {\tt withesis} style. This
%manual is largely based on a similar manual by James Darrell McCauley and Scott Hucker.
%Permission to use, copy, modify and distribute this software and its documentation
%for any purpose and without fee is here by granted.  This software and its documentation
%is provided ``as is'' without any express or implied warranty.

Cloud outage can result in bad user experiences for cloud tenants and revenue loss to the provider. 
This makes cloud network diagnostic solutions invaluable.
Despite the various existing network diagnostic solutions, few of them are designed 
specifically for cloud networks. Current state-of-the-art cloud network diagnosis falls short 
in three aspects: (1) there is no clear way to distinguish whether an observed problem is in the 
tenant's virtual network or in the provider's infrastructure. As a result, 
the interaction between tenants and the provider leads to a longer problem-solving time 
and higher maintenance costs; (2) for cloud tenants, there are only rudimentary troubleshooting 
tools (e.g., ping, VM monitoring) that can be deployed. However, diagnosing a distributed 
system with these tools depends heavily on skill and experience, which is not always 
feasible for tenants; (3) for the cloud provider, new trends such as network function 
virtualization make the infrastructure more complex than the traditional network, 
which could lead to new problems arising. Thus, the provider requires new diagnostic tools to help
cover this range of problems.


\end{abstract}


\clearpage\pagenumbering{arabic} % This makes the page numbers Arabic (1, 2, etc)
