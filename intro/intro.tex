\chapter{Introduction}
\label{chap:intro}
Over the past few years, cloud-based networking solutions have been gaining widespread acceptance and
deployment for organizations such as enterprises and institutions~\cite{paxson1997end}.

\section{Cloud Infrastructure Organization}
% virtual network data plane

%While our work applies to general cloud networking settings, for the purposes of 
In this thesis, we focus on multi-tenant cloud data centers.
In this setting, tenants deploy \emph{virtual private clusters}
composed of application end-points (this could be service software in

\section{Challenges in Diagnosing Cloud Networks}
\label{sec:intro:challenges}
Troubleshooting network problems is difficult. A network is a distributed

\begin{itemize}
\item Cloud networks have higher complexity than traditional networks.  In data planes, network virtualization introduces more software components, including software switches, hypervisors, and so on; software middleboxes are introduced to support network function virtualization. In control planes, the cloud controllers need to perform more functions to set up virtual networks; the controllers map tenants' requirements from a logical view to a physical view, and finally to devices rules. Cloud networks involve a greater number of components involved, and these components may have logical or physical dependency (e.g., exchanging data or sharing hardware resources). This complexity in cloud networks makes them not only more error-prone but also difficulty to manage and diagnose.
\item The two roles in cloud networks makes the management trickier than traditional network. The two roles are the provider (or operator) and the tenants, and their information is isolated from each
other. In the case of network problems, the interaction between the provider and
the tenants is crucial for problem solving; however, this process is usually of low efficiency. 
The tenants observe misbehaviors of their applications directly, but they lack diagnostic tools.
The provider does not know the applications' performance, so they can only wait for tenants' tickets.
%Due to the isolation, both parties cannot perform a complete diagnosis; therefore, 
The two parties need to exchange observations to figure out the problem. This manual process increases
the problem-solving time and maintenance cost.
\item The visibility of cloud networks is not well provided to customers and operators. In traditional networks, network devices and components provide various information for operators to monitor or troubleshoot, e.g., packet drop statistics in switches and the network protocol stack. However, by our study, this kind of visibility is not well preserved in cloud networks. For customers, the cloud provider does not provide visibility of how their packets traverse the network for security reasons. For the provider, some virtualization components are introduced to cloud networks without keeping the visibility for diagnosis\textemdash there are several silent packet drops in VM hypervisors and software middleboxes. This is partially due to the fact that developers typically focus on functionality instead of diagnostic features in the first few versions of the software.
\end{itemize}


\section{An Overview of Our Approach}
\label{sec:intro:approach}
Ever since the birth of networks, network problems have been appearing and upseting network operators. There have been numerous proposals aiming to solve various problems. For example, there are standards or technologies such as SNMP, sFlow, and NetFlow to collect states in networks; there are network models and diagnostic algorithms to discover culprits in network problems; and there are various network diagnostic solutions for traditional networks and software-defined networks, which combine the technologies and algorithms. These approaches are undoubtedly valuable in their specific scenarios.

However, these solutions do not overcome the challenges in diagnosing cloud networks (Section~\ref{sec:intro:challenges}). In this thesis, we make a study of problems in cloud networks. According to the cloud organization, we look into each planes and summarize new problems. In more details, we found that in application planes, the isolation between tenants and the provider causes tenants unable to diagnose their virtual networks, thus we propose, design and implement a virtual network diagnostic service in application planes. We found that in data planes, increasing complexity and lack of visibility cause performance problems difficult to be identified, thus we modify software data planes, collect statistics and leverage the statistics to perform accurate diagnosis. We also find inconsistency issues in control planes, where tenant requirements may not be consistency with physical device states; we leave this problem in future works.

Decomposing cloud networks into three planes and reconsidering diagnostic challenges in cloud networks is an efficient way to discover, abstract and solve cloud network problems. We feel that, owing to our approach, several cloud-specific problems are discovered. Therefore, our works improve the reliability of cloud networks.

\section{Thesis Outline}
\label{sec:intro:outline}
The rest of this thesis is organized as follows. 
In Chapter~\ref{chap:background}, we make a study of cloud network problems and briefly describe our approaches.
In Chapter~\ref{chap:vnd},
we describe our design of the virtual network diagnostic service. In Chapter~\ref{chap:perfsight},
we present our solution for software data plane diagnosis.
In Chapter~\ref{chap:related}, we discuss the related work, 
Finally in Chapter~\ref{chap:conc}, we conclude this thesis and discuss options for future work.
