\chapter{Introduction}
\label{chap:intro}
Over the past few years, cloud-based networking solutions have been gaining widespread acceptance and
deployment for organizations such as enterprises and institutions. 
Today's data centers are infrastructures of various services, including search, content distribution,
big data analysis, and virtual networks. The cloud providers obtain revenue by multiplexing the 
infrastructure among the cloud tenants and the tenants save the trouble of needing to manage the network by themselves.
Recent studies have forecast that global data center traffic (both inter and intra) in 2018 will 
be triple the traffic in 2013~\cite{cisco2018}.

Despite their importance, cloud networks still face the risk of performance, availability and
reliability issues. In April 2011, AWS performed a network change for scaling, 
and shifted the traffic to a low capacity EBS network, which caused a fault whereby the EBS nodes
were unable to find replicas. As such the EBS nodes became ``stuck" in searching 
for replicas \textemdash  the outage lasted for 12 hours!
In April 2013, a misconfiguration with an invalid address for authentication
servers was released to the Google API production environment causing the internal monitoring systems
to become blocked and all of the serving threads to be consumed; as a result, Google API experienced 
a one-hour long outage.

Performance, availability and reliability issues affect the productivity and revenue of the cloud 
providers and tenants. For example, in 2009, AWS experienced an 24-hour outage, which cost a revenue loss of 
about \$4,320,000~\cite{clouddowntime}.
%Theoretically, service outages cost Amazon and Google \$66,240 and \$109,000 per minute, respectively. 
In August 2013, an AWS outage causes its high profile tenants Netflix and 
Airbnb to go down as well.
Motivated by these observations, this thesis proposes two diagnostic solutions\textemdash VND and \Name\textemdash
for cloud network problems. They together form an integral basis of cloud network diagnosis.

In the rest of this chapter, we first describe the cloud infrastructure (Section~\ref{sec:intro:organization}).
Then, in section~\ref{sec:intro:challenges}, we discuss the new challenges for troubleshooting 
in cloud networks compared with the traditional networks. 
Section~\ref{sec:intro:approach} briefly presents our network troubleshooting approaches 
for the new challenges. 
Section~\ref{sec:intro:outline} provides a road map for the rest of thesis.

\section{Cloud Infrastructure Organization}
% virtual network data plane

%While our work applies to general cloud networking settings, for the purposes of 
In this thesis, we focus on multi-tenant cloud data centers.
In this setting, tenants deploy \emph{virtual private clusters}
composed of application end-points (this could be service software in
the case of private data centers, or VMs in the case of public
clouds), network function services (or middeboxes), and logical links
between subsets of them. Network functions improve network performance
or security, and tenants increasingly desire to deploy sophisticated
sets of middlebox functionality within their
clusters~\cite{koponen2014network}.

This setting can be viewed logically as being composed of three planes
(Figure~\ref{fig:arch}): the \emph{control plane}, \emph{application plane}
and \emph{data plane}.  Tenants interact with the application plane,
requesting (re)deployments of virtual private clusters. The control
plane, which the cloud operator runs, responds to such requests by
computing suitable deployment policies, e.g., determining where VMs
and middleboxes ought to be placed, instantiating virtual links
between VMs and middleboxes (using tunneling schemes or encapsulation
policies~\cite{koponen2014network}), and computing the forwarding
state configuration to determine how traffic flows between
VMs/middleboxes and over virtual links. Then the data plane for the tenant's
virtual cluster, where fast path actions are performed on the tenant's
traffic, follows the configurations provided by the control plane
to deliver network traffic between the appropriate end-points in each
virtual cluster.

In this thesis, we focus on middleboxes that are implemented as
software and deployed in VMs attached to virtual switches, which is an
increasingly popular trend also known as \emph{network functions
virtualization (NFV)}~\cite{chiosi2012network}. In NFV, the
middlebox VMs, similar to application VMs, are allocated fixed
resources (e.g., CPU, memory, network bandwidth), and the controller
deploys all VMs to physical machines that have sufficient resources.

Figure~\ref{fig:dp_org_example} shows a tenant with a simple
virtual cluster, consisting of an application VM and a firewall;
furthermore, the tenant requires all Internet traffic to traverse
through the firewall. While this is a simple example, virtual clusters
can have much more complex sequences of middleboxes, where a given
sequence may only apply to a specific traffic substream. 

The physical deployment computed by the controller is shown in
Figure~\ref{fig:dp_org_deploy}. Consider the path that
Internet-originated traffic would traverse to arrive at the tenant VM
(and vice versa for the outgoing traffic), each packet is
forwarded by the cloud gateway to the physical NIC (pNIC) of a
physical server hosting the middlebox VM first; then it traverses the
pNIC driver, the virtual switch, the hypervisor I/O handler, the
virtual NIC (vNIC), the vNIC driver and the VM guest OS network stack,
and finally arrives at the software firewall. After being processed by
the software firewall, the packet traverses all the layers back down
to the pNIC. Then it is delivered by the physical network to the next
hop in the virtual network (a tenant VM or another middlebox).

\section{Challenges in Diagnosing Cloud Networks}
\label{sec:intro:challenges}
Troubleshooting network problems is difficult. A network is a distributed
system, where the states are distributed across devices and software components.
With current rudimentary tools (trace, ping, SNMP, NetFlow, sFlow), troubleshooting
network problems depends heavily on the operators' skill and experience, which
lead to a long problem-solving time and high operational costs~\cite{ndb_thesis}.

In addition to this common difficulty, cloud networks have their own characteristics,
which makes troubleshooting even more difficult. 

\begin{itemize}
\item Cloud networks have higher complexity than traditional networks.  In data planes, network virtualization introduces more software components, including software switches, hypervisors, and so on; software middleboxes are introduced to support network function virtualization. In control planes, the cloud controllers need to perform more functions to set up virtual networks; the controllers map tenants' requirements from a logical view to a physical view, and finally to devices rules. Cloud networks involve a greater number of components involved, and these components may have logical or physical dependency (e.g., exchanging data or sharing hardware resources). This complexity in cloud networks makes them not only more error-prone but also difficulty to manage and diagnose.
\item The two roles in cloud networks makes the management trickier than traditional network. The two roles are the provider (or operator) and the tenants, and their information is isolated from each
other. In the case of network problems, the interaction between the provider and
the tenants is crucial for problem solving; however, this process is usually of low efficiency. 
The tenants observe misbehaviors of their applications directly, but they lack diagnostic tools.
The provider does not know the applications' performance, so they can only wait for tenants' tickets.
%Due to the isolation, both parties cannot perform a complete diagnosis; therefore, 
The two parties need to exchange observations to figure out the problem. This manual process increases
the problem-solving time and maintenance cost.
\item The visibility of cloud networks is not well provided to customers and operators. In traditional networks, network devices and components provide various information for operators to monitor or troubleshoot, e.g., packet drop statistics in switches and the network protocol stack. However, by our study, this kind of visibility is not well preserved in cloud networks. For customers, the cloud provider does not provide visibility of how their packets traverse the network for security reasons. For the provider, some virtualization components are introduced to cloud networks without keeping the visibility for diagnosis\textemdash there are several silent packet drops in VM hypervisors and software middleboxes. This is partially due to the fact that developers typically focus on functionality instead of diagnostic features in the first few versions of the software.
\end{itemize}


\section{An Overview of Our Approach}
\label{sec:intro:approach}
Ever since the birth of networks, network problems have been appearing and upseting network operators. There have been numerous proposals aiming to solve various problems. For example, there are standards or technologies such as SNMP, sFlow, and NetFlow to collect states in networks; there are network models and diagnostic algorithms to discover culprits in network problems; and there are various network diagnostic solutions for traditional networks and software-defined networks, which combine the technologies and algorithms. These approaches are undoubtedly valuable in their specific scenarios.

However, these solutions do not overcome the challenges in diagnosing cloud networks (Section~\ref{sec:intro:challenges}). In this thesis, we make a study of problems in cloud networks. According to the cloud organization, we look into each planes and summarize new problems. In more details, we found that in application planes, the isolation between tenants and the provider causes tenants unable to diagnose their virtual networks, thus we propose, design and implement a virtual network diagnostic service in application planes. We found that in data planes, increasing complexity and lack of visibility cause performance problems difficult to be identified, thus we modify software data planes, collect statistics and leverage the statistics to perform accurate diagnosis. We also find inconsistency issues in control planes, where tenant requirements may not be consistency with physical device states; we leave this problem in future works.

Decomposing cloud networks into three planes and reconsidering diagnostic challenges in cloud networks is an efficient way to discover, abstract and solve cloud network problems. We feel that, owing to our approach, several cloud-specific problems are discovered. Therefore, our works improve the reliability of cloud networks.

\section{Thesis Outline}
\label{sec:intro:outline}
The rest of this thesis is organized as follows. 
In Chapter~\ref{chap:background}, we make a study of cloud network problems and briefly describe our approaches.
In Chapter~\ref{chap:vnd},
we describe our design of the virtual network diagnostic service. In Chapter~\ref{chap:perfsight},
we present our solution for software data plane diagnosis.
In Chapter~\ref{chap:related}, we discuss the related work, 
Finally in Chapter~\ref{chap:conc}, we conclude this thesis and discuss options for future work.
