\chapter{Related Work}
\label{thesis:chapter:related}

{\bf Load Balancing in Datacenters} 
%Load balancing in datacenter networks has been the focus of several studies.
MPTCP~\cite{mptcp,dc-mptcp} is a transport protocol that uses subflows to 
transmit over multiple paths.
CONGA~\cite{conga} and Juniper VCF~\cite{juniper-vcf} both employ congestion-aware flowlet switching~\cite{flowlet} on
specialized switch chipsets to load balance the network.
RPS~\cite{packetspray} and DRB~\cite{drb} evaluate per-packet load balancing on symmetric 1 Gbps networks
at the switch and end-host, respectively.
The CPU load and feasibility of end-host-based per-packet load balancing for 10+ Gbps networks remains open.
%Per-packet load balancing can incur significant end-host overhead for DRB
%if not resorting to jumbo frames.
%Packet reordering problem is not well considered in both RPS and DRB.
%RPS and DRB may perform worse than ECMP in case of topology asymmetry.
Hedera~\cite{hedera}, MicroTE~\cite{microte} and Planck~\cite{planck} use centralized traffic engineering to
reroute traffic based on network conditions.
FlowBender~\cite{flowbender} reroutes flows when congestion is detected by end-hosts and 
Fastpass~\cite{fastpass} employs a centralized arbiter to schedule path selection for each packet.
As compared to these schemes, Presto is the only one that proactively load-balances at line rate for fast networks
in a near uniform fashion without requiring additional infrastructure or changes
to network hardware or transport layers. Furthermore, to the best of our knowledge, Presto is
the first work to explore the interactions of fine-grained load balancing with built-in
segment offload capabilities used in fast networks.
%\textcolor{blue}{~\cite{eden} also advocates to implement network functions such as load balancing 
%at datacenter end hosts.}

{\bf Reducing Tail Latency}
%Detail~\cite{detail} summarizes the causes of long tails of flow completion time---
%1)packet loss and retransmissions,
%2)absence of flow prioritization and 3)uneven load balancing.
%Reducing tail latencies of small mice flows has also been actively studied.
DeTail~\cite{detail} is a cross-layer network stack designed to reduce the tail of flow completion times.
%link layer uses port buffer occupancy to construct lossless fabric,
%network layer performs per-packet adaptive load balancing based on port buffer occupancy,
%transport layer relies upon congestion notifications
%derived from port buffer occupancies,
%finally Detail lets application layer specify flow priorities to
%avoid head-of-line blocking of elephant flows for mice time-sensitive flows.
%Detail modified many layers (including the switch) and is hard to deploy using
%current hardware and network stack.
DCTCP~\cite{dctcp} is a transport protocol that uses the portion of marked packets 
by ECN to adaptively adjust sender's TCP's congestion window to reduce switch buffer occupancy.
%Thus, DCTCP can reduce switch buffer occupancy and reduce flow completion time.
HULL~\cite{hull} uses Phantom Queues and congestion notifications to cap link utilization and prevent congestion.
%HULL uses packet pacing to combat with traffic burstiness in order to leave "bandwidth headroom".
In contrast, Presto is a load balancing system that naturally improves 
the tail latencies of mice flows by uniformly spreading traffic in 
fine-grained units.
%We share the overall goal of reducing mice tail latencies, but instead explore how
%fine-grained load balancing can provide a solution.
QJUMP~\cite{qjump} utilizes priority levels to 
allow latency-sensitive flows to "jump-the-queue" over low priority flows.
PIAS~\cite{bai2015information} uses priority queues to mimic the Shortest Job First principle to reduce FCTs.
%solve the network interference problem caused by elephant and mice flows 
%for datacenter networks. High priority packets are rate-limited 
%at the end-host and can "jump-the-queue" over packets with 
%lower priorities. PIAS~\cite{pias} mimics the Shortest Job First (SJF) 
%principle to reduce flow completion times. It gradually decreases 
%the priority level assigned to a flow based on its flow size.
%Presto is complementary and could be applied on each priority level.
%DIBS~\cite{dibs} detours packets of a congested switch port to a randomly 
%picking neighboring switch to reduce packet drops.
Last, a blog post by Casado and Pettit~\cite{vmware} summarized
four potential ways to deal with elephants and mice, with one advocating
to turn elephants into mice at the edge. 
We share the same motivation and high-level
idea and design
a complete system that addresses many practical challenges of using
such an approach.


{\bf Handling Packet Reordering}
%Many schemes have tried to mitigate the impact of reordering.
TCP performs poorly in the face of reordering, and thus several studies
design a more robust alternative~\cite{rr-tcp,blanton2002making,tcp-pr}.
Presto takes the position that reordering should be handled below TCP in the existing 
receive offload logic.
In the lower portion of the networking stack, SRPIC~\cite{wu2009sorting} sorts reordered packets 
in the driver after each interrupt coalescing event. While this approach can help
mitigate the impact of reordering, it does not sort packets across interrupts, have a 
direct impact on segment sizes, or distinguish between loss and reordering. 
%SRPIC is 
%complementary to our approach because their actions are taken in the driver, before packets
%are pushed to GRO.
%RR-TCP~\cite{rr-tcp} proposed to extend TCP sender to detect and recover from false fast retransmits using DSACK information. 
%As we show, fixing TCP itself cannot solve all the problems incurred by packet reordering.
%\eric{i have some LRO references in the patent slides that we need to make sure are cited somewhere} 
%Instead, Presto's chunking scheme leverages the fact that all the packets going through the same path are in order and has two nice properties:
%1)Presto uses chunkid to make the task of distinguishing packet loss from temporary packet reordering much simpler. 
%2)Presto only needs to make sure the chunks are in order instead of packets, thus reducing per-packet processing overhead.

{\bf Congestion control for DCNs}
%\crs{Rather than proposing a new congestion control algorithm, our work investigates if congestion control can be moved to the vSwitch.
%Thus, many of the following schemes are complimentary.}
DCTCP~\cite{alizadeh2011data} is a seminal TCP variant for datacenter networks.
Judd~\cite{judd2015nsdi} proposed simple yet practical fixes to enable DCTCP in production networks.
TCP-Bolt~\cite{stephens2014practical} is a variant of DCTCP for PFC-enabled lossless Ethernet.
%DCQCN~\cite{zhu2015congestion} is a rate-based congestion control scheme implemented in NICs
%for QCN-based~\cite{qcn} RDMA deployments.
DCQCN~\cite{zhu2015congestion} is a rate-based congestion control scheme (built on DCTCP and QCN) to
support RDMA deployments in PFC-enabled lossless networks.
TIMELY~\cite{mittal2015timely} and DX~\cite{lee2015accurate}
use accurate network latency as the signal to perform congestion control.
TCP ex Machina~\cite{winstein2013tcp} uses computer-generated congestion control rules.
PERC~\cite{jose2015high} proposes proactive congestion control to improve convergence.
ICTCP's~\cite{wu2010ictcp} receiver monitors incoming TCP flows and
modifies~\rwnd{} to mitigate the impact of incast, but this cannot
provide generalized congestion control like~\acdc{}.
Finally, efforts~\cite{dell-toe,chelsio-toe} to
implement TCP Offload Engine (TOE) in specialized NICs are not widely deployed for reasons noted in~\cite{mogul2003tcp,linux-toe}.
vCC~\cite{vcc} is a concurrently designed system that shares~\acdc{}'s goals and some of its design details.
The paper is complementary in that some items not addressed in~\acdc{} are presented, such as a more detailed
analysis of the ECN-coexistence problem, an exploration of the design space, and a theoretical proof of
virtualized congestion control's correctness.~\acdc{} provides an in-depth design and thorough evaluation of
a DCTCP-based virtualized congestion control algorithm on a 10 Gbps testbed.


{\bf Bandwidth allocation} Many bandwidth allocation schemes have been proposed.
Gatekeeper~\cite{rodrigues2011gatekeeper} and EyeQ~\cite{jeyakumar2013eyeq} abstract the network as a single
switch and provide bandwidth guarantees by managing each server's access link.
Oktopus~\cite{Ballani2011oktopus} provides fixed performance guarantees within virtual clusters.
SecondNet~\cite{Guo2010Secondnet} enables virtual datacenters with static bandwidth guarantees.
Proteus~\cite{Xie2012Proteus} allocates bandwidth for applications with dynamic demands.
Seawall~\cite{shieh2011sharing} provides bandwidth proportional to a defined weight by
forcing traffic through congestion-based edge-to-edge tunnels.
NetShare~\cite{Lam2012NetShare} utilizes hierarchical weighted max-min fair sharing to tune relative bandwidth allocation for services.
FairCloud~\cite{Popa2012Faircloud} identifies trade-offs in minimum
guarantees, proportionality and high utilization, and designs schemes over this space.
Silo~\cite{jang2015silo} provides guaranteed bandwidth, delay and burst allowances through a novel VM placement and admission
algorithm, coupled with a fine-grained packet pacer. 
~\acdc{} is largely complimentary to these schemes because it is a transport-level solution.

{\bf Rate limiters}
SENIC~\cite{niranjan2013fastrak}
identifies the limitations of NIC hardware rate limiters (\ie{}, not scalable) and
software rate limiters (\ie{}, high CPU overhead) and uses the CPU to enqueue packets
in host memory and the NIC. Silo's pacer injects void packets into
an original packet sequence to achieve pacing. FasTrack~\cite{niranjan2013fastrak} offloads
functionality from the server into the switch for certain flows.
%~\acdc{} prevents
%TCP flows from sending in the first place and can be used in conjunction with these
%schemes.

