\section{Design}
\label{rate-limiter:sec:design}

\subsection{Direct ECE Marking}
\begin{algorithm}[!t]
\caption{Pseudo-code of Direct ECE Marking Algorithm}
\label{alg:algorithm1}
\begin{algorithmic}[1]
\FOR{each incoming TCP ACK p}
\STATE q $\leftarrow$ rate\_limiter\_queue(p)
\IF{len(q) $>$ {\emph {K}}}
\STATE tcp(p).ece $\leftarrow$ 1
\ENDIF
\ENDFOR
\end{algorithmic}
\end{algorithm}

In this subsection, we introduce a technique called Direct ECE Marking (DEM). 
DEM assumes that VMs and containers are configured with DCTCP congestion control algorithm. 
In~\dem{}, we monitor rate limiter queue occupancy and process each incoming TCP ACK. 
If the current rate limiter queue occupancy is above a threshold $K$, 
we directly set the ACK's TCP ECE (ECN Echo) bit to 1.
To get the correct rate limiter queue occupancy for the TCP ACK, we need to inspect the TCP ACK and
determine which queue the incoming TCP ACK's data packet belongs to. In other words, we need to 
determine the queue that this TCP ACK's reverse flow goes to.
The pseudo-code of~\dem{} is presented in Algorithm~\ref{alg:algorithm1}.~\dem{} can be
implemented in the virtual switch (e.g., OVS) in the hypervisor. OVS rate limiters directly call
the Linux HTB implementation so it can get the rate limiter queue information. Also, OVS processes all the packets
so it can inspect and modify all the incoming TCP ACKs. 

The difference between~\dem{} 
and existing ECN marking schemes is that it directly marks TCP ECE bit based on 
current queue occupancy instead of the queue occupancy one TCP RTT ago if using the existing ECN marking schemes.
Therefore, congestion control actions depend on real-time queueing information and control loop latency 
is reduced to almost 0. Control loop latency is the time it takes to forward the TCP ACK from the 
virtual switch to the VM or container. In this way, ``in network'' latency does not cause 
side-effects for end-host congestion control. Note that if we perform ECN marking on the outgoing path, then 
congestion control loop latency can be very large (e.g., RTT of the flows to remote clients is tens of ms).
Besides reducing control loop latency,~\dem{} also avoids coarse-grained segment-level ECN marking, 
which leads to inaccurate congestion level estimation, as we discussed before.
Therefore,~\dem{} makes rate limiter congestion control more timely and effective.

DEM only turns TCP ECE bit from 0 to 1, it never does the opposite. 
If congestion happens both in the rate 
limiter on the end-host and in the switch(es) on the network path, Then TCP ECE bit is always 1. 
If congestion only happens
in the rate limiter, then~\dem{} turns TCP ECE bit from 0 to 1. 
If congestion only happens in the network (i.e., in the switches), then TCP ECE is kept as 1.
If neither network switches nor the rate limiter is congested, then TCP ECE is always 0.
So~\dem{} does not affect the correctness of end-to-end
congestion control and is complementary with ``in network'' congestion control schemes.

\subsection{~\spring{}}

\begin{algorithm}[!t]
\caption{Pseudo-code of~\spring{} Algorithm}
\label{alg:algorithm3}
\begin{algorithmic}[1]
\FOR{each packet p}
\STATE q $\leftarrow$ rate\_limiter\_queue(p)
\STATE current\_qlen $\leftarrow$ len(q)
\STATE new\_gradient $\leftarrow$ current\_qlen -- q.prev\_qlen
\STATE q.prev\_qlen $\leftarrow$ current\_qlen
\STATE q.gradient $\leftarrow$ (1 -- $\alpha$)*q.gradient + $\alpha$*new\_gradient
\STATE q.normalized\_gradient $\leftarrow$ q.gradient / {\emph {K1}}
\IF{p is an incoming TCP ACK}
\STATE f $\leftarrow$ getReverseFlow(p)
\IF {current\_qlen $<$ {\emph {K1}}}
\STATE f.rwnd $\leftarrow$ f.rwnd + MSS
\ELSIF {current\_qlen $>$ {\emph {K2}}}
\STATE f.ssthresh $\leftarrow$ f.rwnd
\STATE f.rwnd $\leftarrow$ f.rwnd*(1 -- $\beta$*(1 -- $\frac{current\_qlen}{K2}$))
\STATE f.rwnd $\leftarrow$ min(f.rwnd, MSS)
\ELSIF {q.gradient $\le$ 0}
\STATE f.rwnd $\leftarrow$ f.rwnd + MSS
\ELSE
\STATE f.rwnd $\leftarrow$ f.rwnd*(1 -- $\beta$*q.normalized\_gradient)
\STATE f.rwnd $\leftarrow$ min(f.rwnd, MSS)
\ENDIF
\ENDIF
\ENDFOR
\end{algorithmic}
\end{algorithm}

DEM has two limitations. First is that it relies on 
DCTCP transport in VMs and containers. For containers, cloud administrators are able to configure
server's congestion control algorithm to DCTCP. So such an assumption is reasonable. 
However, for VMs, tenants have the flexibility to tune their congestion control settings.
Therefore, assuming that every VM uses DCTCP as the congestion control algorithm is not realistic in practice.
Second,~\dem{} needs ECN support in the network. As mentioned before, 
ECN is not widely supported in WAN traffic~\cite{kuhlewind2013state}.
To address the limitations and make our solution more generic, we present~\spring{} (shown in Algorithm~\ref{alg:algorithm3}).

~\spring{} modifies TCP ACK's receiver's 
advertised window size (RWND) to enforce congestion control~\cite{he2016ac,cronkite2016virtualized}.
It uses real-time rate limiter queue length as congestion control signal and 
a TIMELY-like~\cite{mittal2015timely} congestion control law.
For each packet, 
we get its corresponding rate limiter queue length.
If the packet is outgoing, we get the length of the queue that the packet is to be enqueued.
If the packet is an incoming TCP ACK, we get the length of the queue that 
its reverse flow goes to (TCP is bidirectional).  
We maintain a gradient for the rate limiter queue length using 
Exponentially Weighted Moving Average (EWMA) (line 2--6). 
We set two thresholds, $K1$ and $K2$ ($K1 < K2$). The queue length gradient is normalized by dividing it using $K1$ (line 7).
Note that gradient is a per-queue defined parameter.
If the processed packet is an incoming TCP ACK, we first need to get its reverse 
flow (i.e., the TCP ACK's corresponding data packet flow). Then, 
we manage a running RWND for each flow based on a TIMELY-like congestion control law 
(line 10-- line 20). There are 4 cases: 
if the current rate limiter queue length is smaller than $K1$, that means this is no congestion, so we 
increase the flow's RWND by one MSS (Maximum Segment Size). If the current rate limiter queue length is larger
than $K2$, that means congestion happens in the rate limiter queue, so we multiplicatively decrease the RWND. 
If the current rate limiter queue length is between $K1$ and $K2$, we check the gradient of rate limiter queue occupancy.
If the gradient is smaller than or equal to 0, that means the queue is being drained or its size is not increasing, we 
increase RWND by one MSS. Otherwise, we multiplicatively decrease the RWND based on the normalized gradient.

Note that TIMELY~\cite{mittal2015timely} is a rate-based congestion control algorithm while~\spring{} is a window-based.
TIMELY uses accurate latency measurement provided by the NIC while~\spring{} performs congestion control based on 
real-time rate limiter queue length. Because congestion control decisions are enforced via modifying RWND field in TCP ACK headers,~\spring{} has the following good properties: 
1) the solution does not relies on DCTCP transport in VMs and ECN support in the network, 
so it is generic and can support not only east-west traffic (i.e., intra-datacenter traffic) but also north-south traffic
(i.e., inter-datacenter traffic and traffic between cloud and clients). 
2) the solution avoids coarse-grained segment-level ECN marking and its control loop latency is almost 0, so congestion control
is more effective compared with the strawman solution---DCTCP in VMs/containers and ECN marking in rate limiter queues.  

\subsection{Remarks}
Both~\dem{} and~\spring{} avoid long and unpredictable congestion control loop latency and avoid throughput oscillation due to 
coarse-grained segment-level ECN marking.~\dem{} relies on ECN support in the network and DCTCP transports configured in
the end-points. Compared with~\dem{},~\spring{} is a more generic solution.~\dem{} and~\spring{} share the same limitation, that is
they do not support IPSec (because they need to modify TCP header). However, SSL/TLS is supported. 
Furthermore,~\spring{} needs to maintain per-flow information in the hypervisor. 
Maintaining per-flow information in switches is conventionally considered to be challenging.
In~\spring{} we only need to maintain the information of the connections from the VMs/Containers running on the end-host. 
Also, recent advances like OVS ConnTrack~\cite{ovs-conntrack} has made connection tracking on the end-host more effective.

