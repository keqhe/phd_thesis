\section{Introduction}
\label{sec:introduction}

Bandwidth allocation is an indispensable feature in multi-tenant clouds. 
It guarantees the performance of various applications from multiple tenants running on the same physical server. 
Bandwidth allocation is often implemented by \textit{software rate limiters\footnote{In this paper,
we use software rate limiter and rate limiter interchangeably.}} 
in the operating system (e.g., Linux Hierarchical Token Bucket, aka HTB) due to their flexibility and scalability.
However, rate limiters either uses traffic policing (i.e., dropping packets when
packet arrival rate is above the desired rate) or traffic shaping (i.e., queueing packets in a large
queue to absorb burst and send packets to the network based on token and bucket algorithms).
Thus, bandwidth allocation, low latency and low loss rate can be not achieved at the same time. 
We conduct performance measurements in a public cloud platform (CloudLab~\cite{cloudlab}) and 
find that one of the most widely used software rate limiters, HTB, dramatically increases network latency.

Previous work mainly focused on solving ``in-network'' queueing latency (i.e., latency in 
switches)~\cite{alizadeh2010data,he2016ac,mittal2015timely}, but little research effort has been
done to solve the latency, packet loss and burstiness issues for the software rate limiters on the end-host.
To this end, we explore how we can address the performance issues associated with rate limiters on the end-host in this paper.
Inspired by the efforts to reduce queueing latency in hardware switches, 
we first extend ECN into software rate limiters and configure DCTCP on the end-points. 
However, in our tests, we find simply applying DCTCP+ECN on the end-host causes a problem \textemdash\xspace 
TCP throughput tends to oscillate (between 50\% to 95\% in some cases, see Section~\ref{rate-limiter:sec:measurement}). 
And throughput oscillation would significantly degrade application performance. 

There are two issues with simply applying DCTCP+ECN on end-host rate limiters. The first is that, different from
hardware switches in the network, end-hosts process TCP segments instead of MTU-sized packets. 
TCP Segmentation Offload (TSO)~\cite{tcp-segment-offload} is an optimization technique 
that is widely used in modern operating systems to reduce CPU overhead
for fast speed networks. TSO is enabled in Linux by default. Because of TSO, TCP segment size can reach
64KB in the default setting. That means, if we mark the ECN bits in one TCP segment, a bunch of 
consecutive MTU-sized packets are marked because the ECN bits in the header are copied into each packet in the NIC.
On the other hand, if a TCP segment is not marked, then none of the packets in this segment is marked. This kind of 
coarse-grained segment-level marking leads to an inaccurate estimation of congestion level, and consequently 
leads to throughput oscillation.
The second issue with DCTCP+ECN is that its congestion control loop latency is large. There are three types of 
traffic in datacenter networks \textemdash\xspace 1) intra-datacenter traffic (i.e., east-west traffic) 2) inter-datacenter
traffic and 3) traffic that goes across WANs to remote clients. DCTCP+ECN's control loop latency is one RTT.
But RTT is affected by many factors. For intra-datacenter traffic, RTT is affected by ``in-network'' congestion.
For traffic that goes across WANs to remote clients, its RTT can be tens of milliseconds.
Congestion control loop latency is not bounded and congestion control actions can rely on stale congestion feedback.
The long congestion control loop latency is further exacerbated by the fact that traffic on the end-host is bursty
because of TCP windowing scheme and TSO optimizations~\cite{kapoor2013bullet}.


To address the shortcomings of DCTCP+ECN, 
we design our first mechanism \textemdash\xspace \textit{direct ECE marking (\nameone)}. 
In \nameone, when an ACK is received, \nameone checks the real-time queue length in its corresponding rate limiter, 
and if the queue length exceeds a pre-configured threshold, \nameone \textit{marks the ACK packet instead of data packets}. 
By marking TCP ACK, we mean set TCP ACK's ECN-Echo (ECE) bit.
Directly marking TCP ECE avoids the two shortcomings in DCTCP+ECN. First, the congestion control loop
latency is almost reduced to 0 because it is based on real-time queue length, not the queue length one RTT ago.
Second, it avoids coarse-grained segment-level ECN marking on the transmitting path.
Our design (Section~\ref{rate-limiter:sec:design}) and experiment (Section~\ref{rate-limiter:sec:evaluation}) show that 
DEM eliminates throughput oscillation.
 

There is a prerequisite to deploying \nameone \textemdash\xspace end-hosts are able to react to ECN marking correctly 
(e.g., DCTCP). However, in public clouds, this prerequisite does not always hold 
because the cloud operator does not have access to the network stack configuration in tenants' VMs. 
Even the tenant VM is configured with DCTCP, not all the flows are ECN-capable. For example,
TCP flows between cloud and clients are usually not ECN-capable because ECN is not widely used in WANs~\cite{kuhlewind2013state}.
To make our solution more generic, we adopt the mechanism from AC/DC~\cite{he2016ac}, 
i.e., performing TCP \cwnd computation outside of tenants' VMs and enforcing congestion control decisions via 
rewriting \rwnd. We also design and implement a window-based TIMELY~\cite{mittal2015timely}-like congestion control algorithm. 
We name this mechanism \nametwo. \nametwo completely gets rid of ECN marking and can work for 
both ECN-capable and non ECN-capable flows. 
\nametwo also avoids the two issues that DCTCP+ECN has because for each return ACK, 
we compute \cwnd according to instantaneous rate limiter queue length and queue length gradient and 
rewrite \rwnd in TCP ACK header. 

The contributions of this paper are as follows:
\begin{enumerate}
\item We point out and measure the latency caused by software rate limiters on the end-host in multi-tenant clouds, 
and also show that simply applying DCTCP+ECN is not sufficient to achieve constant bandwidth saturation.

\item We identify the limitations of DCTCP+ECN in end-host networking and 
propose \nameone to solve the throughput oscillation problem.
We also propose \nametwo to make the solution more generic and deployable.

\item We perform a preliminary evaluation to show our solutions can achieve high bandwidth saturation, 
low latency, low oscillation and throughput fairness with negligible CPU overhead.

\iffalse
\item We measure the performance (latency and packet loss) of software rate limiters. We show that software rate limiters can greatly increase end-to-end latency for multi-tenant cloud networks. 
We also show that simply extending ECN in software rate limiter queues and enabling DCTCP on the end-points (i.e., VMs or Container) give sub-optimal performance. 

\item We propose two techniques (\nameone and \nametwo) to enable high throughput, low oscillation, low latency and generic software rate limiters for multi-tenant cloud networks.

\item We evaluate the performance of \nameone and \nametwo. The experiment results demonstrate that the proposed solutions achieve our design goals.
\fi
\end{enumerate}

