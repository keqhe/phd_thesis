\chapter{Related Work}
\label{chap:related}
In this chapter, we discuss various efforts in the area of network diagnosis.
We first discuss the technologies used for network diagnosis. We
then discuss network diagnostic solutions built based 
on these supporting technologies. Finally, we discuss network diagnostic
algorithms that can be used in network diagnostic solutions.


\section{Supporting Technologies}
On end hosts, we have various utilities such as ping,
traceroute and tcpdump~\cite{traceroute, tcpdump}. Ping and traceroute
check reachability in the network, and tcpdump captures
packet traces and prints packet contents.
RFC 4898~\cite{rfc4898} proposes to record extended performance statistics
for TCP, including TCP states and various counters (e.g., bytes, retransmissions).

On switches (both virtual and physical), the industry has standards such as
sFlow, NetFlow and Switched Port Analyzer (SPAN)~\cite{sflow, netflow, span}. 
sFlow samples packets and 
delivers the samples to a server. NetFlow enables a switch to collect statistics
about packets (e.g., IP address, port and protocol) and deliver the statistics to 
a server. SPAN is port mirroring; it enables a switch to duplicate traffic
of one port to another. In the current OpenFlow standard~\cite{openflow}, each switch rule has
its own statistics (e.g., packet count, drops), which helps the network operator
to monitor the routing status.

SNMP~\cite{snmp} is widely used in network management systems to monitor network-attached devices.
The protocol defines the format of the management data between network devices
and the SNMP manager. 

All these technologies provide detailed information in the network. These tools
do not point out problems in the network directly. They rely on the network
operator's experience to judge the problem based on the information.
These technologies usually form the foundation of network diagnostic solutions
(as described in the next section and our VND and \Name).
\section{Network Diagnostic Solutions}
\subsection{Solutions for Network Stacks}
SNAP~\cite{snap} is designed based on RFC 4898. SNAP collects TCP statistics and socket-call logs.
By observing abnormal behaviors of a TCP flow (e.g., delayed ACK), the operator
can narrow down the scope of a fault; by correlating multiple TCP flows on their faulty time,
the operator can find common application misconfigurations.

In X-Trace~\cite{xtrace}, the authors proposed inserting metadata into packets of application tasks. 
The metadata is 
preserved as the packets traverse different network layers. By observing whether
the metadata arrives at each layer successfully, the operator knows whether
a certain layer is functioning correctly.

In NeST~\cite{nest}, the authors proposed instrumenting the network stack. 
States are defined for each component in the network stack based on whether
they are processing or waiting for messages. NeST uses a dependency graph
of components and the components' states to infer the stalled component.

Compared with these solutions, our \Name has two differences. First, PerfSight
is designed for cloud networks instead of traditional networks.
%In the cloud networks, PerfSight monitors more components than the traditional networks. 
%Not only do the cloud networks have more components, but also the components are much more heterogeneous 
The cloud networks have more components and the components are much more
heterogeneous (i.e., blocking/nonblocking middleboxes). 
Second, X-Trace and NeST target the availability issue. They can check whether
a layer/component is processing and delivering packets. \Name targets performance
problems, whose symptoms are not as obvious as availability problems.

\subsection{Solutions for Traditional Networks}
HSA and Anteater~\cite{hsa, anteater} model switch forwarding behaviors and packet headers. 
The model can describe the way in which packets traverse a network device. The authors proposed 
network invariants, such as reachability and loop-free, and used the model
to check these invariants.

Libra~\cite{libra} similarly checks network invariants. It targets networks on a large scale.
In Libra, the authors proposed a map-reduce algorithm to check the network invariants.

Compared with our approaches, VND uses packet traces to perform diagnosis directly,
%provides interfaces that can integrate with these models,
and \Name targets a different scope (i.e., performance problems).
\subsection{Solutions for Software-Defined Networks}
OFRewind~\cite{ofrewind} introduced a proxy between the SDN controller and 
the switches. The proxy records the control messages in the SDN network, and 
provides the replay functionality. By replaying the control messages, the network
problem can reappear, enabling the operator to debug the system.

NetSight~\cite{netsight} is a network-wide wireshark. NetSight collects all packet histories
in the network (optimized by compression), and uses regular expressions as interfaces to filter
packets of interest. The authors designed four applications based on the interfaces. 
The applications check network invariants or profile various network components (e.g., a path or a device).

Frenetic~\cite{frenetic} provides an abstraction layer to operate OpenFlow switches based on the 
OpenFlow syntax. This layer enables the operator to specify packets of 
interest and perform operations such as group and split. 

VeriFlow and NetPlumber~\cite{veriflow, netplumber} are two solutions use to monitor network invariants lively. 
They add a proxy between the SDN controller and the switches in the network. The proxy
intercepts the control messages. Both solutions model the switch forwarding behaviors
and use their models to check network invariants such as reachability, loop-free and
consistency.

NICE~\cite{nice} aims to find bugs in SDN controller applications. It leverages symbolic execution
to perform model checking. During the symbolic execution, the states in the controller
are used to check the network invariants. 

In all these solutions, VeriFlow, NetPlumber and NICE solve problems in the network 
control plane, which is a different scope compared with our works. OFRewind, NetSight and
Frenetic have similarities with our VND approach. Our VND design considers the cloud environment,
and its interfaces retain the abstractions (i.e., a tenant only has views
of its own virtual network). VND's SQL interface provides more operations than NetSight and 
OFRewind, therefore, it simplifies tenants' operations. Compared with Frenetic, VND's view 
comprises is the
entire network instead of a single switch, so the tenant can debug the entire virtual network
comprehensively.

\section{Network Diagnostic Algorithms}
Another set of research literature discusses network diagnostic algorithms.
In these works, the network is usually viewed as a topology composed of nodes and links. 
These works aims to use misbehavior symptoms to locate faulty nodes or links.
These algorithms can be integrated with VND as diagnostic applications, 
but they do not discuss the operational feasibility in virtual networks.
\Name instruments deeply into the software data plane at a finer granularity.
These diagnostics algorithms are as follows.

In netDiagnoser and Max-Coverage~\cite{netdiagnoser, maxcoverage}, the end-to-end path reachability is monitored. The links that 
have the most intersections with the problematic path are suspected to be the root cause links.
In SCORE~\cite{score}, the links in a topology are put into groups if they share the same risk of failure
(e.g., links may share a fiber). SCORE observes the ill-performing links and uses a greedy
algorithm to determine the most suspectable group(s). 
Shrink~\cite{shrink} uses a similar model as SCORE,
but it uses a Bayesian model to infer the root cause group. In~\cite{codebook}, the authors
proposed using a codebook for troubleshooting. The two-dimension codebook has multiple symptoms 
as rows and a set of known problems as columns. Each problem is marked with one or several
symptoms. New problems are matched in the codebook
to find the problem with the most similar symptoms. 
In NetMedic and Sherlock~\cite{netmedic, sherlock}, 
the network topology is viewed
as a Bayesian network, and the faulty probability is computed by historical statistics. In the
case of a network problem, the root case is determined by the probability and the Bayesian network.



