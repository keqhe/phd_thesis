\subsection{Using Embedded Information}
\label{sec-related-embedded}

The idea of using information inside or near the block to detect
errors has been used in several systems. The Cambridge File
Server~\cite{Dion80-CFS} used certain bits in each cylinder (cylinder
map) to store the allocation status of blocks in that cylinder. The
Cedar File System~\cite{hagmann87-cedar} used \textit{labels} inside
pages to check their allocation status. Embedding logical identity of
blocks (inode number + offset) has been done in RAID to recover from
lost and misdirected writes~\cite{KrioukovEtAl08-FAST}. Transactional
flash~\cite{Prabhakaran+08-txflash} embeds commit records inside every
page to provide transactions and recovery. However, NoFS is the first
work that we know of that clearly defines the level of consistency
that such information provides and uses such information alone to
provide consistency.

The design of the Pilot file system \cite{RedellEtAl80-Pilot} is
similar to that of NoFS. Pilot employs self identifying pages and uses
a scavenger to reconstruct the file system metadata upon crash.
However, like the file-system check, the scavenger needs to finish
running before the file system can be accessed. In NoFS, the file
system is made available upon mount, and can be accessed while the
scan is running in the background.

Pangaea~\cite{SaitoEtAl02-Pangaealocal} uses backpointers for
consistency in a distributed wide area file system. However, its use
of backpointers is limited to directory entry backpointers that are
used to resolve conflicting updates on directories. Similar to NoFS,
Pangaea also uses the backpointer as the true source of information,
letting the backpointers of child inodes dictate whether they belong
to a directory or not.

Btrfs~\cite{Mason07-btrfs} supports back references that allow it to
obtain the list of the extents that refer to a particular extent.
Although back references are conceptually similar to NoFS
backpointers, the main purpose of btrfs back references is supporting
efficient data migration, rather than providing consistency. Other
mechanisms such as checksums are used to ensure that the data is not
corrupt in btrfs. Another key difference is that btrfs does not always
store the back reference inside the allocated extent: sometimes the
back references are stored as separate items close to the extent
allocation records.

Backlog~\cite{Macko2010} also uses explicit back references in order
to manage migration of data in write anywhere file systems. The back
references in Backlog are stored in a separate database, and are
designed for efficient querying of usage information rather than
consistency. Backlog's back references are not used for incremental
file-system checking or resolving ownership disputes.

The Selfie virtual disk format~\cite{wu2015selfie} embeds metadata
into data blocks to allow data blocks to be written without requiring
an associated metadata write, thus increasing performance
significantly for data writes. Similar to the non-persistent
allocation structures of NoFS, Selfie's lookup table in maintained in
memory, and reconstructed from the on-disk version after a crash.
While NoFS uses an out-of-band area for storing backpointers, Selfie
depends on the data being compressed to make room for metadata inside
a block. Thus, although Selfie does not depend upon hardware
characteristics, it is dependent on workload characteristics.

The ideas behind ReconFS~\cite{lu14-reconfs} are similar to those of
NoFS. ReconFS makes the directory structure volatile, and recovers it
from the on-disk structure by embedding extra information into the
on-disk structures to make them recoverable. While NoFS maintains both
forward and backward pointers, ReconFS maintains only backward
pointers (inverted indices). NoFS is concerned with the crash
consistency of the whole file system, while ReconFS focuses on the
consistency of the name-space alone. ReconFS and Selfie are
good testaments to our design choices of not persisting certain
metadata structures and allowing I/O to be re-ordered.


