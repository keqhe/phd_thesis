\subsection{Ordered Updates}
\label{sec-related-ordered}

Soft Updates~\cite{GangerPatt-Metadata94} shows how to carefully order
disk updates so as to never leave an on-disk structure in an
inconsistent form. In contrast with OptFS, FreeBSD Soft Updates issues
flushes to implement \sysfsync\ and ordering (although the original
work modified the SCSI driver to avoid issuing flushes).

Given the presence of asynchronous durability notifications, Soft
Updates could be modified to take advantage of such signals. We
believe doing so would be more challenging than modifying journaling
file systems; while journaling works at the abstraction level of
metadata and data, Soft Updates works directly with file-system
structures, significantly increasing its complexity.

OptFS is similar to that of Frost \etal's work on
Featherstitch~\cite{Frost+07-GenFSDep}, which provides a generalized
framework to order file-system updates, in either a soft-updating or
journal-based approach. OptFS instead focuses on delayed ordering for
journal commits; some of our techniques could increase the journal
performance observed in their work.

Lu \etal's work on loose-ordering consistency~\cite{cmutechreport2011,
  lu2014loose} has a similar flavor to OptFS. They increase the
performance of writes to persistent memory by allowing speculative
writes and only making the writes visible at transaction commit. The
dirty status of writes in the cache are identified using extra tag
bits in the cache.

Apart from file systems, write order is important in a different (but
related) context. A number of storage systems today use SSD or RAM
devices as caches to increase performance~\cite{holland2013flash,
  saxena2012flashtier}. For such storage systems, the policy
controlling how data is transferred from the cache to the backing
device is crucial for performance and reliability. The setup in these
systems is similar to the buffer cache and the disk storage device.

Koller \etal\ propose writeback policies (ordered write-back and
journaled writeback) to ensure consistency at the storage level
(either point-in-time or epoch-based)~\cite{koller2013write}. Qin
\etal\ take advantage of the fact that applications do not have
consistency expectations in between two barrier events (\eg \sysfsync)
to further increase performance~\cite{qin2014reliable}. OptFS
writeback is similar to the journaled writeback policy of Koller
\etal, with \sysfsync\ calls or time-duration triggers defining epochs.

Write ordering is also important in non-volatile memory systems.
Recent work by Pelley \etal\ ~\cite{pelley2014memory} explores how
memory consistency models relates to persistent writes in non-volatile
memory, and how the write orders may be relaxed while maintaining
consistency.
