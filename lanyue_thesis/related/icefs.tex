\section{File System Isolation}

Isolation is an important property in modern systems. 
Various isolation techniques are proposed for different parts of the 
system.  Typical examples include virtual
machines~\cite{bugnion97disco,DragovicEtAl03-Xen}, Linux Containers~\cite{linux-container}, 
isolation kernel~\cite{WhitakerEtAl02-Denali}, BSD {\tt
  jail}~\cite{Poul+00-Jail} and Solaris zones~\cite{Solaris11-Zones}.  
They are used to create an independent environment for the target
applications or even operating systems to run in a shared platform.  
For the file system component, these frameworks only provide namespace
isolation for different clients by constraining clients to only a
subset of the shared file system.  However, as shown in this thesis,
the failure, recovery and journaling performance are not isolated in
a shared file system.  Our work is to provide a file system level
container which can be used together with the above techniques to
provide full isolation for applications.

IceFS has derived inspiration from a number of projects for improving
file system recovery and repair, and for tolerating system crashes.

Many existing systems have improved the reliability of file systems
with better recovery techniques.  Fast checking of the Solaris UFS~\cite{PeacockEtAl98-SolarisFsck} has been proposed by only
checking the working-set portion of the file system when failure
happens.  Changing the I/O pattern of the file system checker to reduce
random requests has been suggested~\cite{BinaEmrath89-FasterFsck,Ma+13-ffsck}.
A background fsck in BSD~\cite{McKusick02-BackgroundFsck} checks a file system 
snapshot to avoid conflicts with the foreground
workload. WAFL~\cite{HitzEtAl94-WAFL} employs Wafliron~\cite{WAFL-Iron}, an
online file system checker, to perform online checking on a volume but the
volume being checked cannot be accessed by users.  
% AD removed for space -- return it if you want it
Our recovery idea is based on the cube abstraction which provides isolated
failure, recovery and journaling.  Under this model, we only check the
faulty part of the file system without scanning the whole file system.
The above techniques can be utilized in one cube to further speedup
the recovery process.

Several repair-driven file systems also exist. 
Chunkfs~\cite{HensonEtAl06-Chunkfs} does a partial check of Ext2 by
partitioning the file system into multiple chunks; however, files and
directory can still span multiple chunks, reducing the independence of
chunks.  Windows ReFS~\cite{Windows-ReFS} can automatically recover
corrupted data from mirrored storage devices when it detects checksum
mismatch. Our earlier work~\cite{LuEtAl13-HS} proposes a high-level
design to isolate file system structures for fault and recovery
isolation. Here, we extend that work by addressing both reliability
and performance issues with a real prototype and demonstrations for
various applications.  
% AD removed for space
Compared with these file systems, IceFS disentangles the file
system into both logically and physically isolated parts.  In this
systematical manner, the fault detection and recovery can be more
independent and localized.

Many ideas for tolerating system crashes have been introduced at
different levels.  Microrebooting~\cite{CandeaEtAl04-Reboot}
partitions a large application into rebootable and stateless
components; to recover a failed component, the data state of each
component is persistent in a separate store outside of the
application.  Nooks~\cite{SwiftEtAl03-Nooks} isolates failures of
device drivers from the rest of the kernel with separated address
spaces for each target
driver. Membrane~\cite{SundararamanEtAl10-Membrane} handles file 
system crashes transparently by tracking resource usage and  
the requests at runtime; after a crash, the file system is restarted
by releasing the in-use resources and replaying the failed requests.  
The Rio file cache~\cite{Chen96-Rio} protects the memory state of the
file system across a system crash, and conducts a warm reboot to
recover lost updates.  
Inspired by these ideas, IceFS localizes a file system crash by
microisolating the file system structures and microrebooting a cube
with a simple and light-weight design.
Address space isolation technique could be used in cubes for better
memory fault isolation.   

In addition to reliability isolation, performance isolation within
file systems and SSDs also have been proposed. When running multiple
workloads on a machine with fast storage devices and many cores, the
contention of in-memory locks from different threads may introduce a
large overhead. Similar to cubes in IceFS,
Multi-lane~\cite{Kang14-multilane} and SpanFS~\cite{spanfs}  
propose to isolate I/O stacks (both in-memory and on-disk structures)
for different domains; in this manner, domains will not compete for
shared locks, avoiding lock scalability bottlenecks when running
multiple workloads in a many core machine.
Similar to metadata entanglement in file systems, data with different
lifetime could be stored together in the same flash page in modern
SSDs, leading to excessive garbage collection traffic. 
Multi-streamed SSDs~\cite{multi-stream} propose streams for
applications, mapping data with different lifetimes to SSD streams.
The multi-streamed SSD ensures that the data in a stream are not only
written together to a physically related flash space, but also
separated from data in other streams; thus, the SSD throughput and
latency QoS can be significantly improved. Mapping cubes to different
streams may improve performance of IceFS when running on such
multi-streamed SSDs. 
 



