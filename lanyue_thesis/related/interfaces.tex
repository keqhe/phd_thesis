\section{Interfaces for Consistency and Durability}
\label{sec-related-interfaces}

We now describe interfaces used in various systems to achieve
consistency and durability.

\vspace{0.1in} \noindent \textbf{Transactions}. Several file systems
have provided a transactional interface, allowing developers to update
application state with ACID guarantees~\cite{Olson07-NTFS,
  PorterEtAl08-TxOS, spillane2009enabling, seltzer1990transaction}.
However, either due to the performance cost (14\% in
TxOS~\cite{PorterEtAl08-TxOS}), or high
complexity~\cite{Olson07-NTFS}, transactional file systems have not
been adopted widely.

Several storage devices provide support for the transactional
interface~\cite{Prabhakaran+08-txflash, choi2009jftl, park2005atomic,
  galtoledo05-flash, lu2013lighttx, kang2013x,
  ouyang+11-beyondblockio, coburn2013aries, lu15-blurred}.
Applications or file systems can build on top of these devices to
allow the user to atomically update application
state~\cite{min2015lightweight}.

\vspace{0.1in} \noindent \textbf{Atomic Update}. Several recent
systems offer an interface for users to obtain atomicity (without
isolation)~\cite{fusionIOmariaDB, verma2015failure, park2013failure}.
Park \etal\ offer an atomic version of
\texttt{msync()}~\cite{park2013failure}, while Verma \etal\ provides
failure-atomic \texttt{writev()}, \texttt{msync()}, and
\texttt{syncv()}~\cite{verma2015failure}. The MariaDB database builds
on top of the atomic write offered by FusionIO
SSDs~\cite{fusionIOmariaDB}.

At the application level, atomic update is usually carried using the
\texttt{rename()} system call~\cite{URLPosix-Rename}. There are subtle
problems with using \texttt{rename()}, depending upon the file-system
configuration~\cite{Thanu+14-Alice, URLmassivefsthread}. Specialized
interfaces such as \texttt{exchangedata()}~\cite{Apple-exchangedata}
and \texttt{replacefile()}~\cite{Windows-ReplaceFile} are also used
to achieve atomic updates.
 
\vspace{0.1in} \noindent \textbf{Barriers and Flushes}. The \syscsync\
and \sysdsync\ primitives are similar to the fence and flush memory
primitives in the Intel architecture~\cite{Intel2010-Arch}. The key
difference is that a \sysdsync\ issues a flush at the end of
\syscsync, and thus is a superset of \syscsync; on the other hand,
flushes do not guarantee the ordering provided by barriers.
Mnemosyne~\cite{Volos+11-Mnemosyne}, a lightweight system for exposing
storage-class memory to user-mode programs, builds upon the Intel
primitives to offer the persistent-memory version of flush and fence
primitives to users (among other interfaces such as transactions).
Since Mnemosyne is only concerned with data in processor caches and
storage-class memory, Intel's primitives are sufficient, and no
additional hardware support is required. In contrast, since current
disks do not offer the fence primitive, hardware modifications like
asynchronous durability notifications are required for OptFS.

BPFS~\cite{Condit+09-BPFS} is another file system built on top of
phase-change memory. BPFS introduces two new hardware primitives:
8-byte atomic writes and epoch barriers. Similar to \syscsync, epoch
barriers order writes without affecting durability. Note that while
BPFS uses epoch barriers internally, it does not present an interface
such as \syscsync\ to the user. While OptFS requires modifications to
the storage device, BPFS requires modifications to the processor core,
the cache, the memory controller, and the phase-change memory chips.

Featherstitch~\cite{Frost+07-GenFSDep} provides similar primitives for
ordering and durability: \texttt{pg\_depend()} is similar to
\syscsync, while \texttt{pg\_sync()} is similar to \sysdsync. The main
difference lies in the amount of work required from application
developers: Featherstitch requires developers to explicitly
encapsulate sets of file-system operations into units called
\emph{patchgroups} and define dependencies between them. Since
\syscsync\ builds upon the familiar semantics of \sysfsync, we believe
it will be easier for application developers to use.

We believe that barriers and flushes are the most fundamental of these
interfaces; other interfaces can be implemented using barriers and
flushes. Mnemosyne~\cite{Volos+11-Mnemosyne} builds transactions and
atomic updates using barriers and flushes; it also exposes the
low-level primitives to users so that they can build their own
consistency mechanisms. We advocate a similar approach for
applications using file systems to store state.
