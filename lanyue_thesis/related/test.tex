\section{Testing File Systems}
\label{sec-related-test}

There have been several efforts to test the reliability of storage
systems. We discuss the work most closely related to BOB
(\sref{chap-bob}).

Zheng \etal\ developed a framework for testing whether commercial
databases violate ACID properties under power failure
scenarios~\cite{ZhengEtAl14-Torturing}. They trace the writes resulting
from one of four hand-crafted workloads, and replay different
sequences of those writes to detect ACID violations. Their framework
differs from BOB in two significant ways. First, they do root-cause
analysis on ACID violations to help engineers diagnose the bug. Since
BOB was not developed for the purpose of diagnosing bugs, it does not
contain tools for root-cause analysis. Second, their framework is
limited to SCSI devices (as they modify the iSCSI driver); BOB, on the
other hand, works above the device-driver level, and therefore works
on any storage device.

% Explode
The EXPLODE framework~\cite{YangEtAl06-Explode} systematically checks
storage systems for errors. It drives the system into rare corner
cases and tests system behavior. EXPLODE is a powerful framework that
can be used to test applications, file systems, and software at any
level of the storage stack for errors. However, for the simple task
for testing persistence properties of file systems, we feel it is
needlessly complex. For example, EXPLODE requires developers to
carefully annotate complex file systems using \textit{choose()} calls.
BOB, on the other hand, does not require any annotation or specialized
knowledge of the file system being tested.

% dm-power-fail
% not testing explicitly persistence properties
There has been interest in the Linux kernel community towards more
effective power-failure testing. Josef Basik, a software engineer at
Facebook, has been developing a tool based on the device mapper to
easily reproduce power failures~\cite{Basik-power-patch,
  Basik-power-testing-lwn}. Similar to BOB, the tool logs writes and
then creates new disk images based on the traces. However, the focus
of the tool is on whether file systems properly make acknowledged
writes durable, and not on testing persistence properties.

Zheng \etal\ have tested the reliability of SSDs and disk drives under
power failure~\cite{ZhengEtAl+13-ssd}. Unlike the other work discussed
here, they developed hardware to cut power to the devices being tested
and examined device behavior. Hardware testing is essential for
testing storage devices, since the firmware is closed-source, and we
cannot emulate the effects of power loss, as we do for software layers
in the storage stack. Thus, Zheng \etal's work is complementary to
tools such as BOB.

We note that none of the tools discussed in this section examine or
test the persistence properties of file systems. File-system
persistence behavior was largely unexplored before our work, and BOB
represents the first step in defining and examining persistence
properties.  
