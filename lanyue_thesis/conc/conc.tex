
\section{Conclusion}

In this dissertation, we present three research projects --- 1) an edge-based traffic 
load balancing system (i.e., Presto) for datacenter networks, 2) virtualized congestion 
control technique for multi-tenant clouds (~\acdc TCP) and 3) low latency software rate 
limiters for cloud networks. All of them leverage the flexibility and high 
programmability of software-defined network edge (i.e., end-host networking) to improve 
the performance of datacenter networks. Each of these three projects focuses on one 
crucial functionality in datacenter networks. In the following, we provide a short 
conclusion for each project in turn.


{\bf Presto: Edge-based Load Balancing for Fast Datacenter Networks.} 
Modern datacenter networks are built with multi-stage Clos networks. There are 
usually tens to hundreds of network paths between two servers in the same datacenter.
The state-of-the-art traffic load balancing uses flow-level schemes (e.g., ECMP and WCMP). 
However, flow-level load balancing schemes suffer from the elephant flow collision problem. 
Elephant flow collisions lead to reduced throughput and increased latency. 
We propose Presto to address this classic problem. Presto is an end-host-based traffic 
load balancing system. At the sender side, Presto uses virtual switch (OVS) to 
chunk elephant flows into flowcell (a flowcell consists of multiple consecutive 
TCP segments and its maximum size is 64KB) and spread the flowcells evenly over 
multiple network paths. At the receiver side, we design and implement improved Generic 
Receive Offload (GRO) functionality in Linux networking subsystem to mask packet reordering for TCP. 
Presto makes sure mice flows (smaller than or equal to 64KB in size) are not exposed to packet 
ordering issue. Note that in realistic datacenter networks, the overwhelming majority of the 
flows are mice flows. For elephant flows, Presto's improved GRO logic puts flowcells in 
order before they are pushed up to TCP. Presto eliminates the elephant flow collision 
problem and demonstrates that subflow-level traffic load balancing is possible and effective. 


{\bf ~\acdc{} TCP: Virtual Congestion Control Enforcement for Datacenter Networks.}
 In multi-tenant clouds, tenants manage their own Virtual Machines (VMs). VM TCP stacks' congestion 
control algorithms can be outdated, inefficient or even misconfigured. Those outdated, inefficient 
or even misconfigured VM TCP stacks can cause severe network congestion and throughput fairness issue. 
Network congestion and throughput unfairness affect the performance of the applications running in 
clouds (e.g., increased task completion time). To address this problem, we present~\acdc{} TCP, 
a virtual congestion control enforcement technique for datacenter networks. The key idea of~\acdc{} TCP 
is to implement an intended congestion control algorithm (i.e., DCTCP) in the virtual switch in the hypervisor 
and the congestion control decisions are enforced via modifying TCP header's~\rwnd{} field. 
Our experiment results show that enforcing an intended congestion control algorithm in 
the network virtualization layer can greatly reduce network latency and throughput fairness. 
Also,~\acdc{} TCP's CPU and memory overhead is negligible.  

{\bf Low Latency Software Rate Limiters for Cloud Networks.}
 Rate limiters are employed to provide bandwidth allocation feature in multi-tenant clouds. 
For example, in Google Cloud Platform, different kinds of VMs are allocated with 
different maximum bandwidth. However, we find that rate limiters can increase network latency 
by an order of magnitude or even higher. That is because traffic shaping 
(the underlying mechanism of rate limiters) maintains a queue to absorb bursty traffic 
and dequeues packets into the network based on preconfigured rate. To solve this 
problem, we first extend ECN into rate limiter queues and apply DCTCP on the end-host. 
Though this straightforward scheme reduces network latency significantly, it can also 
lead to TCP throughput oscillation because of coarse-grained segment-level ECN marking 
and long congestion control loop latency. To address the shortcomings of the straightforward scheme, 
we present~\dem{}, which directly sets TCP ECE bit in reverse ACKs and~\spring{} which runs 
a queue-length-based congestion control algorithm and enforces congestion control decisions 
via modifying~\rwnd{} field in reverse ACKs. Both~\dem{} and~\spring{} enable low latency, 
high network saturation rate limiters.~\dem{} relies on ECN support while~\spring{} is 
generic and can handle both ECN flows and non-ECN flows.

The techniques and mechanisms proposed in this dissertation are original. 
We believe the research work presented in this dissertation will be valuable to 
the computer networking research community.

%\cite{example}
