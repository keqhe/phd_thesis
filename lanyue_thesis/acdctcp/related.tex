\section{Related Work}
\label{related}
This section discusses different classes of related work.

\tightparagraph{Congestion control for DCNs}
\crs{Rather than proposing a new congestion control algorithm, our work investigates if congestion control can be moved to the vSwitch.
Thus, many of the following schemes are complimentary.}
DCTCP~\cite{alizadeh2011data} is a seminal TCP variant for datacenter networks.
Judd~\cite{judd2015nsdi} proposed simple yet practical fixes to enable DCTCP in production networks.
TCP-Bolt~\cite{stephens2014practical} is a variant of DCTCP for PFC-enabled lossless Ethernet.
%DCQCN~\cite{zhu2015congestion} is a rate-based congestion control scheme implemented in NICs
%for QCN-based~\cite{qcn} RDMA deployments.
DCQCN~\cite{zhu2015congestion} is a rate-based congestion control scheme (built on DCTCP and QCN) to
support RDMA deployments in PFC-enabled lossless networks.
TIMELY~\cite{mittal2015timely} and DX~\cite{lee2015accurate} 
use accurate network latency as the signal to perform congestion control.
TCP ex Machina~\cite{winstein2013tcp} uses computer-generated congestion control rules.
PERC~\cite{jose2015high} proposes proactive congestion control to improve convergence.
ICTCP's~\cite{wu2010ictcp} receiver monitors incoming TCP flows and 
modifies~\rwnd{} to mitigate the impact of incast, but this cannot
provide generalized congestion control like~\acdc{}.
Finally, efforts~\cite{dell-toe,chelsio-toe} to 
implement TCP Offload Engine (TOE) in specialized NICs are not widely deployed for reasons noted in~\cite{mogul2003tcp,linux-toe}.
%~\acdc{} is designed to work with commodity NICs. 

\tightparagraph{Bandwidth allocation} Many bandwidth allocation schemes have been proposed.
Gatekeeper~\cite{rodrigues2011gatekeeper} and EyeQ~\cite{jeyakumar2013eyeq} abstract the network as a single
switch and provide bandwidth guarantees by managing each server's access link.
Oktopus~\cite{Ballani2011oktopus} provides fixed performance guarantees within virtual clusters.
SecondNet~\cite{Guo2010Secondnet} enables virtual datacenters with static bandwidth guarantees.
Proteus~\cite{Xie2012Proteus} allocates bandwidth for applications with dynamic demands.
Seawall~\cite{shieh2011sharing} provides bandwidth proportional to a defined weight by
forcing traffic through congestion-based edge-to-edge tunnels. 
NetShare~\cite{Lam2012NetShare} utilizes hierarchical weighted max-min fair sharing to tune relative bandwidth allocation for services.
FairCloud~\cite{Popa2012Faircloud} identifies trade-offs in minimum
guarantees, proportionality and high utilization, and designs schemes over this space.
Silo~\cite{jang2015silo} provides guaranteed bandwidth, delay and burst allowances through a novel VM placement and admission 
algorithm, coupled with a fine-grained packet pacer. As discussed in~\sref{background}, 
~\acdc{} is largely complimentary to these schemes because it is a transport-level solution.

\tightparagraph{Rate limiters} 
SENIC~\cite{niranjan2013fastrak} 
identifies the limitations of NIC hardware rate limiters (\ie{}, not scalable) and 
software rate limiters (\ie{}, high CPU overhead) and uses the CPU to enqueue packets 
in host memory and the NIC. Silo's pacer injects void packets into 
an original packet sequence to achieve pacing. FasTrack~\cite{niranjan2013fastrak} offloads
functionality from the server into the switch for certain flows.~\acdc{} prevents
TCP flows from sending in the first place and can be used in conjunction with these
schemes.


\tightparagraph{Low latency DCNs}
Many schemes have been proposed to reduce latency in datacenter networks.
HULL~\cite{alizadeh2012less} uses phantom queues to leave bandwidth headroom to support low latency.
pFabric~\cite{alizadeh2013pfabric} is a clean-slate
design which utilizes priority and minimal switch buffering to achieve low latency.
Fastpass~\cite{perry2014fastpass} uses a centralized arbiter to
perform per-packet level scheduling.
QJUMP~\cite{qjump} uses priority queueing and rate limiting to
bound latency. Traffic engineering~\cite{al2010hedera,rasley2014planck} and 
load balancing~\cite{alizadeh2014conga,he2015presto,ghorbani2015micro} can also
reduce latency. Because~\acdc{} works on the transport level, it is
largely complimentary to these works.

\tightparagraph{Performance-enhancing proxies}
Several schemes improve end-to-end protocol performance via a middlebox
or proxy~\cite{RFC3449,RFC3115,balakrishnan2008maelstrom,davern2011httpep,balakrishnan1995improving}.
\acdc{} fits into this class of works, but is unique in providing a mechanism
to alter a VM's TCP congestion control algorithm by modifying the vSwitch.

\tightparagraph{Virtualized congestion control}
\crs{vCC~\cite{vcc} is a concurrently designed system that shares~\acdc{}'s goals and some of its design details.
The paper is complementary in that some items not addressed in this work are presented, such as a more detailed
analysis of the ECN-coexistence problem, an exploration of the design space, and a theoretical proof of
virtualized congestion control's correctness. Our paper provides an in-depth design and thorough evaluation of
a DCTCP-based virtualized congestion control algorithm on a 10 Gbps testbed.
}
