
\svnidlong{$LastChangedBy$}{$LastChangedRevision$}{$LastChangedDate$}{$HeadURL: http://freevariable.com/dissertation/branches/diss-template/frontmatter/abstract.tex $}
\vcinfo{}

% Don't change anything above this line. It must stay in exactly the
% same spacing.


Digital data is essential to our daily lives and businesses. 
The amount of data generated each year grows exponentially. Various
storage systems were developed to meet different data management
requirements. For decades, researchers and practitioners have been
trying to overcome a central challenge in these storage systems: how
to reliably and efficiently access data on different storage devices.  

In the first part of the dissertation, we respond to this challenge
with a comprehensive study of modern Linux file systems to understand 
practical reliability and performance problems in existing file
systems. We analyze eight years of Linux file-system changes across
5079 patches. We focus on file-system bugs by studying their
fine-grained patterns, consequences and trends. We find that about
40\% of patches are bugs, existing in both new and mature
file systems; they do not diminish despite the stability. Most of the
bugs lead to system crashes or data corruption. We also exam
performance and reliability patches. The performance techniques used
were relatively common and widespread. About a quarter of performance 
patches reduced synchronization overheads. Reliability techniques
seemed to be added in a rather ad hoc fashion. 

From the file system study, we find that file systems lack enough
reliability isolation: a small fault can impact the whole file system.
To solve this problem, we propose IceFS, a novel file system that
separates physical structures of the file system for better isolation.
A new abstraction, the cube, was provided to enable the grouping of
files and directories inside a physically isolated 
container.  We show three major benefits of cubes within IceFS:
localized reaction to faults, fast recovery, and concurrent file-system
updates. We demonstrate that IceFS is able to localize failures that
were previously global, and recover quickly using localized online or
offline fsck. IceFS can also provide specialized journaling to meet diverse
application requirements for performance and consistency.
Furthermore, we conduct two cases studies where IceFS is used to
host multiple virtual machines and is deployed as the local file
system for HDFS data nodes. IceFS achieves fault isolation and fast
recovery in both scenarios, proving its usefulness in modern storage 
environments.

Motivated by physical separation techniques leading to better
performance in IceFS, we continue to explore similar techniques in
another important type of storage systems: key-value stores. 
We present WiscKey, a persistent LSM-tree-based key-value 
store with a novel performance-oriented data layout that separates
keys and values to minimize I/O amplification. The data layout and I/O
patterns of WiscKey are highly optimized for SSD devices. 
We solve a number of reliability and performance challenges
introduced by the new key-value separation architecture. We propose a
parallel range query design to leverage the SSD's internal parallelism
for better range query performance on unordered datasets. We also
introduce an online and light-weight garbage collector for WiscKey to
reclaim the invalid key-value pairs without affecting the foreground
workloads much. We demonstrate the advantages of WiscKey with both
microbenchmarks and YCSB workloads. Microbenchmark results show that
WiscKey is 2.5$\times$ - 111$\times$ faster than LevelDB for loading a
database and 1.6$\times$ - 14$\times$ faster for random lookups.
WiscKey is faster than both LevelDB and RocksDB in all six YCSB
workloads, and follows a trend similar to the microbenchmarks. 

