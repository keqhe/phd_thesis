
\svnidlong{$LastChangedBy$}{$LastChangedRevision$}{$LastChangedDate$}{$HeadURL: http://freevariable.com/dissertation/branches/diss-template/frontmatter/abstract.tex $}
\vcinfo{}

% Don't change anything above this line. It must stay in exactly the
% same spacing.

Datacenter networks are critical building blocks for modern cloud computing 
infrastructures. 
In this dissertation, we show how we can leverage 
the flexibility and high programmability of datacenter network edge 
(i.e., end-host networking)~\cite{ovs-extending,pfaff2015design} to 
improve the performance of three key functionalities in datacenter networks --- traffic load balancing, 
congestion control and rate limiting.

Datacenter networks need to deal with a variety of workloads, ranging
from latency-sensitive small flows to bandwidth-hungry
large flows. In-network hardware-based load balancing schemes which are based on flow hashing,
e.g., ECMP, cause congestion when hash collisions occur. 
We propose a soft-edge load balancing scheme called Presto.
Presto load-balances on near uniform-sized small data units (flowcells) and 
spread flowcells across the symmetric network via the virtual switches on the senders.
Because of fine-grained flowcell-level load balancing, packets may arrive out of order
at the receiver side, so propose a mechanism to handle reordering in the 
Generic Receive Offload (GRO) functionality below the TCP layer.
Presto avoids the hash collision problem and improves traffic load balancing performance significantly.

Optimized traffic load balancing alone is not sufficient to 
guarantee high-performance datacenter networks. 
Virtual Machine (VM) technology plays an integral role in
modern multi-tenant clouds by enabling a diverse set of software to be run
on a unified underlying framework. This flexibility, however,
comes at the cost of dealing with outdated, inefficient,
or misconfigured TCP stacks implemented in the VMs. 
We propose a congestion control virtualization technique called AC/DC TCP.
AC/DC TCP exerts fine-grained control over arbitrary tenant
TCP stacks by enforcing per-flow congestion control in
the virtual switch (vSwitch) in the hypervisor. AC/DC TCP is light-weight,
flexible, scalable and can police non-conforming flows. 

Besides queueing latency in network switches, we observe that 
rate limiters on end-hosts can also increase network latency by an order of magnitude or even more. 
To this end, we propose two techniques \textemdash\xspace~\dem{} and~\spring{} to 
improve the performance of rate limiters.
Our experiment results demonstrate that~\dem{} and~\spring{}-enabled
rate limiters can achieve high stable throughput and low latency.


