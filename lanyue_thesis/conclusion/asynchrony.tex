\subsection{The Importance of Asynchrony}
\label{sec-conc-lessons-asynchrony}

The benefits of asynchrony have long been known in the software
world~\cite{michelson2006event, AdyaEtAl02-Tasks}. For example, when
asynchronous Javascript was introduced, it enabled the creation of
exciting, responsive web applications such as Google Suggest and Google
Maps~\cite{garrett2005ajax}.

Yet, the role of asynchrony in storage systems has been limited. Until
the 1990s, storage systems were essentially synchronous, performing
one operation at time~\cite{Seltzer90-SchedRevisit,
  McKusickEtAl-FFS-84}. With the introduction of tag queuing in SCSI
disks~\cite{AndersonEtAl03-SCSIvATA, RidgeField00-SCSI, Weber04-SCSI},
disks could accept sixteen simultaneous requests. Unfortunately, many
devices do not implement tag queuing
correctly~\cite{marshall2012disks}.

In this dissertation, we have shown that asynchronous, orderless I/O
has significant advantages. First, not constraining the order of I/O
allows large performance gains, especially on today's multi-tenant
systems~\cite{thereska2013ioflow}. Recent work on non-volatile memory
has shown that removing ordering constraints on I/O can increase
performance by 30$\times$~\cite{pelley2014memory}. 

Second, using interfaces such as asynchronous durability notifications
allows each layer to introduce optimizations such as delaying,
batching, or re-ordering I/O, without affecting the correctness of the
file system or the application. Increasing the independence of each
layer in the storage stack leads to a more robust storage system.
