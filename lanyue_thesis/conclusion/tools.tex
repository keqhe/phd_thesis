\subsection{The Need for Better Tools}
\label{sec-conc-lessson-tools}

During the course of this dissertation work, we realized that there is
a need for better frameworks for analyzing, testing, and verifying
file-system crash consistency. Despite decades of research on file
systems, there are few open-source tools available for inspecting
file-system consistency.

\vspace{0.1in} \noindent \textbf{Tools for Analyzing Crash
  Consistency}. While working on Probabilistic Crash Consistency, we
had to build a framework to analyze the different ordering
relationships that need to hold for a journaling file system to remain
consistent in the event of a crash. Analyzing these relationships
eventually led us to discover Selective Data
Journaling~\cite{Chidambaram+13-OptFS}, the precise circumstances in
which not journaling data blocks provides the same crash-consistency
guarantees as when data blocks are journaled. Specifically, if the
data block is not already part of a file, journaling it does not
provide any additional crash guarantees. Selective Data Journaling
improves performance significantly for workloads where large files are
updated atomically via mechanisms such as \texttt{rename()}(\eg in
text editors such as Gedit~\cite{gedit}).

In multiple interactions, researchers and developers in the
file-system community have reported that the Selective Data Journaling
idea was obvious once explained. However, this insight has eluded
file-system researchers and practitioners for many years (although it
is known in the software transactional memory
community~\cite{capturedmem09, Harris06, adl2006compiler}). We believe
this is due to two reasons: the inherent complexity of protocols like
journaling, and the complexity associated with implementations such as
ext4~\cite{newcombe2015amazon}. Without using a framework to analyze
the guarantees provided by these systems and the requirements for
those guarantees, even simple insights are hard to obtain. Just as
practitioners in distributed systems formally specify the design of
their systems and verify optimizations~\cite{newcombe2015amazon,
  Lamport02-BookTLA}, we believe a similar effort is required in the
realm of file systems.

\vspace{0.1in} \noindent \textbf{Tools for Testing Crash Consistency}.
While building the No-Order File System and the Optimistic File
System, we needed to test the reliability of the file system. We found
that there were no open-source tools available to inject either
targeted failures or random failures. As a consequence, we had to
develop our own tools for this purpose. We have received requests from
other universities to share these tools. Although the current versions
of these tools are specific to NoFS and OptFS, we believe generalized
versions of these tools would be greatly aid file-system development
and research. We strongly advocate for better tools that aid
file-system research to made open-source (\eg the Filebench
suite~\cite{Filebench05}).
 
\vspace{0.1in} \noindent \textbf{Tools for Verifying Crash
  Consistency}. Modern storage stacks are complex and consist of many
layers~\cite{thereska2013ioflow}. Crash consistency of the application
on top of the stack depends on every layer between the application and
storage handling requests such as flushes and FUA correctly.
Configuration options at each layer (\eg journaling mode of file
system) affects the crash consistency of the application.

We believe that storage stacks have grown so complex that it is not
feasible for humans to observe the composition of a storage stack and
reason about whether a given application will obtain crash consistency
on top of that stack. Furthermore, with customized on-demand
software-defined storage~\cite{vmware-sds, emc-sds, netapp-sds,
  vmware-sdspaper}, a human verifying a composed storage stack is not
practical.

We believe that the design of each layer (the aspects that relate to
crash consistency) must be formally defined, and that we should
develop tools to automatically verify the crash guarantees that a
storage stack claims to provide. We have undertaken initial efforts in
this direction~\cite{Alagappan+15-verification}.
  
