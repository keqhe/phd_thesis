\subsection{The Importance of Interface Design}
\label{sec-conc-lessons-interfaces}

In this dissertation, we studied existing crash-consistency problems
in file systems and applications. Most of these problems can be
directly traced to bad interfaces provided by storage devices and file
systems.  

\vspace{0.1in} \noindent \textbf{Storage Interface}. Storage devices
at the lower end of the market (such as SATA and IDE drives) expose
the flush command for managing the disk cache. There are two problems
with the flush interface: coarseness, and lack of visibility. First,
the flush interface is too coarse: there is no option to selectively
flush a few blocks to non-volatile storage. If the cache is filled
with dirty data, out of which the file system only wants to make a few
blocks durable, the current interface forces a lot of inefficient
waiting. Second, file systems do not have visibility into the
durability status of blocks in the disk cache. Combined with the
coarseness of the flush interface, this leads to a lot of un-necessary
waiting for the file system.   

\vspace{0.1in} \noindent \textbf{File-System Interface}. File systems
implement the POSIX interface~\cite{posix2013} that includes the
\sysfsync\ system call for making dirty writes durable. The \sysfsync\
call results in a flush command at the storage level, and therefore
inherits the performance problems associated with the flush interface.
As mentioned in Section~\ref{sec-tmodel}, the \sysfsync\ interface
couples together ordering and durability; applications desiring only
ordering between writes are forced to pay the cost of durability. 

These interface flaws have led to widespread problems: disk drives
that lie about flushing~\cite{RajimwaleEtAl11-CCE, URLMacFsync},
system administrators who deliberately put their systems at risk for
performance~\cite{URLmassivefsthread}, systems that do not honor flush
requests~\cite{virtualbox-ignoreflush}, and finally, application
developers who do not use \sysfsync\ fearing the performance
cost~\cite{tso-fsync}.

% open interfaces

\vspace{0.1in} \noindent \textbf{The importance of open interfaces}.
We believe that the closed nature of storage-device interfaces have
resulted in significant additional complexity and loss of performance.
Much of file-system research can be traced back to working around the
pitfalls of badly-designed interfaces. We believe that making the
device interface open will greatly aid the storage community. Several
groups are working towards this with open-channel
SSDs~\cite{shin2015providing, ouyang2014sdf, wang2014efficient}.

