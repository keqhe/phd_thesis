% The intro

Storage has becoming increasingly important to our modern world. 
Data is generated in an unprecedented speed, and new storage hardware
is also on the rise. Under this situation, various storage 
systems and applications have been developed to manage data more
reliably and efficiently. In this dissertation, we presented study
results that help us to understand the real problems in file systems
and proposed solutions that offer better reliability and performance
than existing file systems and key-value stores. 

We started by analyzing a large number of file system patches to
understand what are the real problems in modern file systems (\sref{chap-fs}).  
We focused on file system bugs, performance and reliability patches. 
We found that there are many bugs in both new and mature file systems,
and most of bugs may lead to serious consequences (data corruption or
system crashes). However, there is not enough isolation within a file
system to cope with corruption, crashes and performance problems. 
Therefore, we presented our solution IceFS, a new file system that
separates physical structures of the file system for better
reliability and performance isolation (\sref{chap-icefs}).  In
addition to file systems, we also found that physical entanglement of
keys and values in LSM-trees can lead to large I/O amplification and
disappointed performance in fast SSDs. We proposed WiscKey, a novel
LSM-tree based key-value store with a performance-oriented data layout
that separates keys and values to minimize I/O amplification(\sref{chap-wisckey}). 

In this chapter, we first summarize our contribution of this
dissertation (\sref{sec-conc-summary}). We then describe a set of  
general lessons we have learned in the course of this dissertation
work (\sref{sec-conc-lessons}). Finally, we discuss several possible
future research directions (\sref{sec-conc-future}). 

\section{Summary}
\label{sec-conc-summary}

This dissertation is mainly comprised of two parts. In the first part,
we analyzed and studied file-system patches to understand the real problems 
of modern file systems. In the second part, we built a file system and
a key-value store with physical separation techniques for better
reliability and performance. We summarize each part in turn.


\subsection{File System Study} 
\label{sec-conc-study}

In the first part of the dissertation, we performed the first
comprehensive study of the evolution of Linux file systems.
First, we investigate the overview of file-system patches. We found
that nearly half of total patches are for code maintenance and
documentation. The remaining dominant category is bugs, existing in
both new and mature file systems. Interestingly, file-system bugs do
not diminish despite the stability.  We also found that bug patches
are generally small while feature patches are significantly larger. 

Second, we further analyzed bugs in detail. We found that semantic
bugs are the dominant bug category (over 50\% of all bugs), which are
hard to detect via generic bug detection tools. Concurrency bugs are
the next most common (about 20\% of bugs), more prevalent than in
user-level software. The remaining bugs are split relatively evenly
across memory bugs and improper error-code handling. Unfortunately,
most of the bugs we studied lead to crashes or corruption, and hence
are quite serious. We also made an important discovery that nearly
40\% of all bugs occur on failure-handling paths. 

Finally, we also studied performance and reliability patches. The
performance techniques used were relatively common and
widespread. About a quarter of performance patches reduced
synchronization overheads. Reliability techniques seemed to be added
in a rather ad hoc fashion. Inclusion of a broader set of reliability
techniques could harden all file systems.  

\subsection{Physical Separation in Storage Systems}
\label{sec-conc-icefs}

In the second part of this dissertation, we presented our new
physical separation techniques in two important types of storage
systems: file systems and key-value stores. By building IceFS and
WiscKey, we demonstrated that physical separation is a useful and
practical technique, which can lead to significantly better
reliability and performance for various workloads and environments. 

First, we proposed IceFS, a novel file system that separates physical
structures of the file system for better isolation.  A new
abstraction, the cube, was provided to enable the grouping of files
and directories inside a physically isolated container. 
To realize disentanglement, IceFS was built upon three core principles:
no shared physical resources, no access dependencies, and no bundled
transactions among cubes. IceFS ensured that files and
directories within cubes are physically distinct from files and
directories in other cubes; thus data and I/O within each cube is
disentangled from data and I/O outside of it. 

We showed three major benefits of cubes within IceFS: localized
reaction to faults, fast recovery, and concurrent file-system updates. 
We showed how cubes enable localized micro-failures; crashes and
read-only remounts that normally affect the entire system are now
constrained to the faulted cube. We also showed how cubes permit
localized micro-recovery; instead of an expensive file-system wide
repair, the disentanglement found at the core of cubes enables 
IceFS to fully (and quickly) repair a subset of the file system (and even
do so online). In addition, we illustrated how transaction splitting
allows the file system to commit transactions from different cubes in
parallel, greatly increasing performance (by a factor of 2$\times$ - 5$\times$) for 
some workloads.  Furthermore, we conducted two cases studies where
IceFS is used to host multiple virtual machines and is deployed as the
local file system for HDFS data nodes. IceFS achieved fault isolation
and fast recovery in both scenarios, proving its usefulness in modern
storage environments.

Second, we presented WiscKey, an SSD-conscious persistent key-value
store derived from the popular LSM-tree implementation, LevelDB. The
central idea behind WiscKey was the separation of keys and values;
only keys were kept sorted in the LSM-tree, while values were stored
separately in a log. This simple technique can significantly reduce
write amplification by avoiding the unnecessary movement of values
while sorting. Furthermore, the size of the LSM-tree was also
noticeably decreased, leading to better caching and fewer device reads
during lookups. WiscKey retained the benefits of LSM-tree technology,
including excellent insert and lookup performance, but without
excessive I/O amplification. 

We solved a number of reliability and performance challenges
introduced by the new key-value separation architecture.
First, range query (scan) performance may be affected because values
are not stored in sorted order anymore.  We proposed a parallel range
query design to leverage the SSD's internal parallelism for better range
query performance on unordered datasets. 
Second, WiscKey needed garbage collection to reclaim the free
space used by invalid values. We introduced an online and light-weight
garbage collector for WiscKey to reclaim the invalid key-value pairs
without affecting the foreground workloads much. We demonstrated the
advantages of WiscKey with both 
microbenchmarks and YCSB workloads. Microbenchmark results showed that
WiscKey is 2.5$\times$ - 111$\times$ faster than LevelDB for loading a
database and 1.6$\times$ - 14$\times$ faster for random lookups;
similar results hold for YCSB workloads. 


\section{Lessons Learned}
\label{sec-conc-lessons}

In this section, we present a list of general lessons we learned while
working on this dissertation.  

%lessons of file-system study
\vspace{0.1in} \noindent \textbf{Large-scale studies are valuable.} 
In general, studies drive system designs.  Researchers and
practitioners conducted numerous studies in the past to help
understand system behaviors, workload patterns, and various
design tradeoffs.  These detailed studies can provide practical
motivation and useful design guidelines for next generation systems. 

For the file system study project, we found many interesting and
important details after studying a large number of patches across six
file systems. These details can inspire new research opportunities. For 
example, once we understand how file systems leak their resources,
we can build a specialized tool to detect them more efficiently. Once
we know how file systems crash, we can improve current systems to
tolerate them more effectively. These vivid examples in the study
really teach us what are the important problems in file systems.  

The scale of the study is also very important.  More patches we
studied, more interesting patterns we found. More importantly, only
after a large number of cases are studied, we can make some
observations which may be statistically significant and insightful. 

\vspace{0.1in} \noindent \textbf{Conquer a large-scale study with small steps.}
Conducting a large-scale study is definitely time consuming. 
The total number of patches we studied comprehensively is
about 2000 (bugs, performance and reliability patches). If we knew in
advance that we need to analyze 2000 hard patches, we may feel
overwhelmed and give up this project early on. The way we handled this
study is starting from small beginnings. 

Initially, we were just curious about Ext3's patches, since Ext3 was a
stable and popular file system. We studied 5 versions of Ext3 patches,
and classified them into different categories. We found the results
very interesting and surprising.  Then, we thought that we
need to study more patches of Ext3 to better understand the broader
patterns.  Once we finished all 40 versions of Ext3 in Linux 2.6
series, we wondered that whether Ext4 has similar bugs and performance
techniques as Ext3. After Ext4, we continued to ask that what about 
other types of file systems, such as Btrfs (copy-on-write) and XFS
(logical journaling).  In this incremental manner, we grew our study
base from one file system to six popular file systems, and from less
than 100 patches to over 5000 patches in total finally.  It took us
one and half year to finish the study. Thus, a large-scale study is
still feasible and manageable.  Starting from small beginnings and
making consistent progress keep us interested and mentally sane in
this long journey.  


\vspace{0.1in} \noindent \textbf{Research should match reality.}
Most previous work of bug studies and bug-finding tools focused on
generic bugs (such as memory bugs, concurrency bugs and error handling
bugs). However, in our study, we found that a majority of file system
bugs are semantic bugs, which require file-system domain knowledge to 
understand and fix.  Furthermore, we found that none of these semantic
bugs was detected by existing tools, such as Coverity.  This striking
gap between research and reality demonstrates the importance of
finding and solving real problems. 

In our study, we advocated that new tools are highly desired for
semantic bugs in file systems. We also suggested that more attention
may be required to make failure paths more correct.  Fortunately,
following research papers~\cite{MinEtAl15-SOSP,YuanEtAl14-OSDI}
proposed solutions for these two real problems. 


\vspace{0.1in} \noindent \textbf{History repeats itself.} 
We observed that similar mistakes happened again and again, within a
single file system and across different file systems. Developers of
new file systems even borrowed bug-fixing solutions from patches of
old file systems. We also found that similar performance and
reliability techniques were used across file systems in
different time when performance bottlenecks were detected or data
corruption occurred. 

System researchers and developers should not only focus on innovative
designs for future systems, but also respect the history of existing
systems widely deployed and researched. Learning from these rich
history will never be wasting of time; instead, it will provide new
insights and experiences of what worked and what not. More
importantly, the same mistakes can be avoided at the first place. 
We should pay more attention to system histories, learn from them, and
build a correct, high-performance and robust next generation systems
from the beginning. 

%lessons of icefs
\vspace{0.1in} \noindent \textbf{Inspiration can come from a different area.}
After we finished the file system study, we were looking for what to
work next.  During that period, we occasionally read a security paper
from our colleague~\cite{VaraEtAl12-CCS}, which solved the problem of 
performance interference among virtual machines co-located in the same
physical machine in the cloud.  Since there was little isolation
for VMs from different users, it is possible to slow down other VMs
co-located in the same machine by running a carefully design workload
from the attacker VM. 

Even though this is a security paper, it immediately inspired us to 
think about isolation for VMs or users on the same machine in the
context of file systems. Since we just finished the file system study,
we knew that file systems have many bugs, and bugs cause data
corruption and system crashes, which will affect all VMs or users
relying on the same file system. An interesting research question for
us is that how can we isolate reliability interference in the file
system layer. This is how we started the IceFS project.  Later, we
also extended IceFS to handle the performance interference in the file
system by isolating transactions. 

Ideas from a different area may help you think about your research in
a new perspective. Viewing problems in a different angle is hard
without any new source of inputs. These small and random kicks from
other research areas may inspire you in unexpected ways and play an
important role for new research ideas.  


\vspace{0.1in} \noindent \textbf{Don't settle for existing abstraction.}
Files and directories are basic and long-standing abstraction provided
by file systems.  However, these logical entities are an illusion; the
underlying physical entanglement in file system data structures and
transactional mechanisms does not provide true isolation. 
To provide true isolation within a file system, we proposed a new
abstraction, called cube.  This new abstraction connects the logical
abstraction provided to users and the underlying physical structures
on disk. Based on this new abstraction, it is straightforward for us
to design an isolation file system which can provide both reliability
and transaction performance isolation.  

New abstraction fosters new research. If a research problem cannot be 
solved by existing abstraction, a new abstraction may be required. 
New abstraction should be simple and easy to use. 

\vspace{0.1in} \noindent \textbf{Isolation should be a fundamental design goal.}
When we analyzed the shared failures and bundled performance in file
systems, we found the root cause is the entanglement of on-disk  
structures and in-memory transactions for different files. In other
words, file systems were not designed to provide isolation at the
first place.  To provide better data locality, metadata from multiple
files is stored together in the same disk block. To provide better I/O
performance, updates from different files are batched in the same
transaction. Data locality and I/O performance were the main goals
when designing file systems.  However, isolation was omitted as a
fundamental design goal. 

Isolation is becoming more important in new environments. As the
workloads of the world move to the cloud, as the computing moves to
virtual machines and containers, as the multi-tenant world becomes the
only world we will live in, isolation is the key property to give us
the illusion that we have our own machines. We should rethink systems 
underneath our applications at the very basic levels, both in terms of 
data layouts and I/O patterns. We should design systems with strong
isolation for well-defined boundaries from the beginning. 

\if 0
\vspace{0.1in} \noindent \textbf{Don't miss the big system stack.}
Modern systems in data centers have a large number of stackable layers
due to virtualization, compression, replication, deduplication,
etc. For example, Windows I/O stack has 18 layers between applications and the
device~\cite{thereska2013ioflow}. 

We should think any system problem with the whole system stack in
mind.  
\fi 


%wisckey
\vspace{0.1in} \noindent \textbf{Don't put old software in new hardware.}
Originally, LSM-trees were designed for machines with hard drives and 
a small number of cores. As long as the write and read amplification
are smaller than 1000, LSM-trees are good enough for a wide range of
workloads.  With the rise of SSDs on modern servers, while replacing
an HDD with an SSD underneath an LSM-tree does improve performance,
the SSD's true potential goes largely unrealized with the old
LSM-trees as we demonstrated in Chapter~\ref{chap-whiskey}. 

We need to evolve old software for new hardware. There are lots of
progresses on building storage systems in last several decades.
Virtually, all of those intelligence was based on hard drives. 
Recently, we really transition our storage system to flash based
devices. We should re-evaluate systems designed before, leverage
things worked well, and optimize them further to leverage the new
hardware. In WiscKey, we leverage the good parts of LSM-trees 
(such as sequential I/O patterns and rich features), and
further optimize it in new ways for SSDs to get the best of two
worlds. 

\vspace{0.1in} \noindent \textbf{Work on systems extremely slow or unreliable.}
At the conference Usenix FAST 2009, Marshall Kirk McKusick said that
the reason he worked on FFS (Fast File System)~\cite{McKusickEtAl-FFS-84} was 
that the default UNIX file system that time only utilized 2\% of the device 
bandwidth. There was a huge room to improve it. Therefore, FFS was
proposed and it can reach 47\% of the device bandwidth, more than
20$\times$ of the baseline. 

WiscKey is also such an example. When our experiments showed that LevelDB
has high I/O amplification, and it can only utilize about 1\% of the
SSD device bandwidth, we felt that it is a great opportunity to make
LevelDB significantly faster. After an array of new designs and
optimization, WiscKey can be over 100$\times$ faster than LevelDB.    

We believe that when choosing what to work on, try to choose an
existing system which is extremely slow or unreliable. More
opportunities lie at these corners.  




\section{Future Work}
\label{sec-conc-future}

Non-volatile memory devices are at the rise.  NAND-flash based solid state disks
(SSD)~\cite{Caulfield+09-Gordon,Grupp+09-FlashMeasure}, phase-change
memory (PCM)~\cite{Caulfield+10-Moneta,Condit+09-BPFS} and
memristor~\cite{Strukov08-memristor} provide microsecond level latency
and high internal data parallelism, which can greatly boost
application performance.  This revolution of storage technologies is 
transforming the state-of-art of the storage hardware and software.  

However, the entire storage stack and many key-value stores are
designed for an ancient technology: the classic (and slow) hard
drive~\cite{CardEtAl94-Ext2,Iyer01-Anticipatory,Kleiman86-Vnodes,McKusickEtAl-FFS-84,  
Seltzer90-SchedRevisit,WorthingtonEtAl94-Scheduling}.
Various designs and architectures are based on assumptions of slow I/O
bottleneck in the system.  Simply replacing hard disks with fast SSDs
will probably not achieve the full performance benefits with current
system software.  

In addition, commodity servers contain an increasing number of computing cores.   
Servers with tens to hundreds of cores are available already~\cite{seamicro}.
Trends indicate that the number of cores within a single machine will
continue to increase in future~\cite{Borkar07-ThousandCore}.  The
cache and memory capacity also increases with the number of cores for
balanced performance; it is not uncommon that a single server machine
contains over 100 GB of DRAM for high performance~\cite{Clements+13-Commute,David+13-Sync}. 

We believe these two hardware trends (fast storage devices and many
cores) will continue in foreseen future.  Our vision is to build
highly scalable storage stack and applications.  In this section, we
discuss several directions for such vision. 

\subsection{Scalable Virtual File System (VFS)}

VFS is the entrance of all system calls.  All the generic file system 
structures are maintained in VFS, such as inode, super block, and dentry.
VFS also maintains metadata and file data caches for fast accesses:
inode cache, dentry cache and page cache.  When reading or writing a file,
the related cached structures also need to be updated.  
We are interested in exploring several potential scalability
bottlenecks in VFS. 

First, VFS uses spin locks to protect the global
inode hash table and dentry LRU list.  Frequent insertion and deletion
of inodes or directory entries may trigger these synchronization
bottlenecks.  To solve this challenge, we propose to decompose the
global shared structures into multiple smaller ones.  A new partition
domain (e.g., a similar abstraction as cube) can be introduced in VFS
to isolate these generic structures; thus, the big lock can be avoided.  

Second, directories in file systems are organized as a tree
hierarchy both logically and physically.  To access a
file {\tt /home/bob/research/paper/foo}, all the directory entries
from the root to the target file must be parsed.  
During the directory traversal, if multiple threads access
different files in deep directories, then all the dentries and paths
along the directory path will be frequently locked and released.  This
style of directory hierarchy dependency may affect scalability across
many cores.  Furthermore, if each directory contains many entries,
then the traversal process may require a fair number of 
I/Os.  Even though efficient lookup structures, such as Btree, are
helpful to reduce the overhead of lookup, the fundamental bottleneck
still exist because of the directory hierarchy. 

We should decouple the logical hierarchy and the physical layout of
directories for better scalability.  We propose to build a directory
hierarchy over a object-based store.  In this manner, given a
pathname, the file system only needs to lookup the corresponding
object on disk, instead of parsing all the entries along the path.

\subsection{Scalable Local File Systems}

Many popular Linux file systems, such as Ext3, Ext4 and XFS were
designed decades ago. The scalability of local file systems on  
new hardware is also very interesting to explore. 

First, file systems use a wide range of synchronization methods to
protect their internal metadata, such as spin locks, read/write locks
and mutexes.  These lock primitives are also used extensively in
kernel for shared data structures.  Developers make constant efforts
to optimize existing locks for better
concurrency~\cite{LuEtAl13-fsstudy}, 
such as removing unnecessary locks, using finer-grained locking
instead of big locks, and replacing write locks with RCU.  

However, these techniques do not consider the scalability issue on
many cores.  Typical locks used in Linux may not be scalable across 
cores due to the cache coherence protocol~\cite{David+13-Sync}.
It may be the worth effort to adopt more complex and scalable locks in
file systems~\cite{Mellor91-Locks}.  Furthermore, we could change
current locking granularity for better 
concurrency.  For example, to update a file, the inode lock is
required for exclusive accesses.  Two threads may update different
pieces of metadata or data of the same file concurrently.  Thus,
fine-grained locking may be more scalable for certain workloads.  

Second, transaction management in file systems can cause significant
performance degradation for certain workloads.  For example,
Ext3/4 maintain only one running transaction, and use one single
thread to flush the buffered transaction to disk periodically or 
as triggered by {\tt fsync()}.  For a multiple-thread write intensive 
workload, the journaling layer can block the applications due to the
serialization in the transaction layer.  

We propose to parallelize the journaling layer for better scalability.
First, we need to maintain multiple running transactions in memory to
buffer independent updates.  As a result, updates from 
applications should be classified in some way to be independent from
each other.  Second, during commit, we will use multiple threads to
commit transactions in parallel.  Third, a shared physical journal or
multiple physical journals should both work, since the bottleneck is
not supposed to be at the device level. 


Third, storage device locality may be irrelevant, if there is little
difference between random and sequential performance on fast devices.
However, most existing allocation or scheduling algorithms are
optimized for data locality, which could limit the concurrency of file
systems.  For example, to allocate data blocks for a file in Ext3 or
Ext4, the block group of its parent directory is preferred.  If
multiple threads allocate blocks for files under the same parent
directory, a shared bitmap will be updated from multiple cores,
limiting concurrency.  

We propose randomized algorithms to replace traditional locality based
algorithms in file systems.  A randomized allocation design can spread
the metadata updates uniformly across the whole device.  We will
revisit all the locality-based algorithms in file systems, and replace
them for better scalability if possible. 

\subsection{Scalable Block Layer}
The block layer is below the local file system. It is responsible for
scheduling the low level I/O requests to the storage device.  We are
interested in two scalability issues in the block layer.  

First, I/O schedulers usually store many pending requests in queues
before dispatching them, such as CFQ and Deadline.  One potential
bottleneck is the shared request queue, which is used to buffer all
the incoming I/O requests.  The shared queue lock is required when the
block layer does request insertion, request merging, fairness
scheduling, I/O accounting, and request deletion.  This single point
processing could be a scalability bottleneck for fast
devices~\cite{Bjorling+13-SSDSched}.  We are curious to further
explore other structures and algorithms in the block layer which can
slow down I/O request processing.  

Second, the block layer also sends device cache flush requests from
file systems to the underlying device drivers.  For example, a 
{\tt fsync()} request from an application could force a device cache
flush for durability of its data.  However, cache flush is
expensive, and this could cause slowdown for applications.  For a
highly parallel application, there may be many cache flush requests
from different threads.  We are interested in investigating new
techniques to smartly schedule these cache flush requests for both
durability and scalability.  

\subsection{Scalable Key-Value Stores}

LSM-trees were designed for machines with hard drives, and a small
number of cores. In WiscKey, we optimize LSM-trees for SSDs by
separating keys and values. There are many other aspects we can
further improve WiscKey for SSDs, large memory and many cores. 

In WiscKey, the garbage collection is done by a single background
thread.  The thread reads a chunk of key-value pairs from the tail of
the vLog file; then for each key-value pair, it checks the LSM-tree
for validity; finally, the valid key-value pairs are written back to
the head of the vLog file. We can improve the garbage collection in
two ways. First, lookups in the LSM-tree are slow since multiple
random reads may be required.  To speedup this process, we can use
multiple threads to do the lookup concurrently for different key-value 
pairs. Second, we can make garbage collection more effective by
maintaining a bitmap of invalid key-value pairs in the vLog file. When
the garbage collection is triggered, it will first reclaim the chunk
with the highest percentage of free space. 

Another interesting direction to scale LevelDB or WiscKey is
sharding the database. Many components of LevelDB are single-threaded
due to a single shared database.  As we discuss before, there is a
single memtable to buffer writes in memory. When the memtable is full,
the foreground writes will be stalled until the compaction thread
flushes the memtable to disk.  In LevelDB, only a single writer can be
allowed to update the database. The database is protected by a global
mutex. The background compaction thread also needs to grab this mutex
when sorting the key-value pairs, competing with the foreground
writes.  For a multiple writer workload, this architecture can
unnecessarily block concurrent writes.  One solution is to partition
the database and related memory structures into multiple smaller
shards. Each shard's keys will not overlap with others. Under this
design, writes to different key-value ranges can be done concurrently
to different shards.  A random lookup can also be distributed to one
target shard, without searching all shards.  This new design may make
lookups faster because of a smaller dataset to search.  

\section{Closing Words}
\label{sec-conc-closing}

Digital data is universal and essential in many aspects of our
lives and businesses. With the increasing amount of data generated and
the rise of new hardware, accessing data both reliably and efficiently
become critical for modern storage systems.  

In this dissertation, we began our journey with a comprehensive study
of Linux file systems evolution, to better understand the reliability
and performance problems that plague existing systems for
decades. Then, we demonstrated that physical separation of fundamental 
data structures in file systems and key-value stores can provide
isolated reliability and significantly better performance.  This is
extremely important for our changing storage world, which becomes
virtualized, multi-tenant and failure-prone. We hope that this
dissertation can serve as a simple but detailed example for
researchers and system builders to rethink the fundamental data
layouts and I/O patterns of existing systems, leverage past valuable
experiences, and embrace new hardware and optimization for a better 
next generation storage system.

